
%\usepackage[latin1]{inputenc}
\usepackage[utf8]{inputenc}
%\usepackage[french]{babel}
\usepackage[T1]{fontenc}
\usepackage{amsmath}
%\usepackage{amsfonts}
%\usepackage{amssymb}
\usepackage{makeidx}
\usepackage{graphicx}
\usepackage{lmodern}
\usepackage[left=2.5cm,right=2.5cm,top=2cm,bottom=2cm]{geometry}  % marges
\usepackage{color}
\usepackage{xcolor}
\setcounter{secnumdepth}{4} % augmentation de la num?rotation des sous-sections
\setcounter{tocdepth}{3} % augmentation de la profondeur de la table des mati?res

%\setlength{\parindent}{0cm} %Para las sangrias

% Ce package je ne sais pas pour qu'il soit utilisé
\usepackage{titlesec}

% Ce package est pour plusiers rangées
\usepackage{multirow}

% Ce package est pour la représentation des molécules. Il a besoin de chemist.sty et assurechemist.sty
\usepackage{chemist}

% Ce package est pour la représentation des nombres et des unités de mesure
%\usepackage{si}

\usepackage{svg}

\usepackage{bm}

% Ce package est pour écrire les références avec natbit dans le Harvard style
%\usepackage{natbib}
%\addbibresource{files/bibliographyFile.bib}


% Ce package est pour écrire les références avec biblatex: biblatex.sty, etoolbox.sty, logreq.sty, logreq.def, url.sty
%\usepackage{biblatex}
%\usepackage{cite}
%\addbibresource{bibliographyFile.bib}
%\DeclareNameAlias{sortname}{family-given}
%\DeclareNameAlias{default}{family-given}

% Ce package fait que les phrases ou expressions changent de ligne
\usepackage{makecell}

% Pour les hyperreferences
\usepackage[hidelinks]{hyperref}

% Pour changer l'orientation des foies
\usepackage{pdflscape}
\usepackage{lscape}
%\usepackage[paper=portrait,pagesize]{typearea} % Careful! Changes all the document into portrait mode!

% Pour emphatiser des equations (tables ...)
%\usepackage{empheq}

% Pour changer localement les margins
\usepackage{changepage}

%%%%%%%%%%%%%%%%%%%%%%%%%%%% Pour la bibliographie %%%%%%%%%%%%%%%%%%%%%%%%%%%%%%%

% Bibliographie avec natbib:
\usepackage{natbib}
\bibliographystyle{agsm} %agsm, unsrtnat

\usepackage{subcaption}

%% Bibliographie avec biblatex
%\usepackage[backend=bibtex]{biblatex}
%\addbibresource{files/bibliographyFile.bib}
%\DeclareNameAlias{sortname}{family-given}
%\DeclareNameAlias{default}{family-given}

%%%%%%%%%%%%%%%%%%%%%%%%%%%%%%%%%%%%%%%%%%%%%%%%%%%%%%%%%%%%%%%%%%%%%%%%%%%%%%%%%%


\newcommand{\citeColor}[1][]{{\color{blue}\cite{#1}}}
\newcommand{\citepColor}[1][]{({\color{blue}\citealt{#1}})}
\newcommand{\citemColor}[1][]{({\color{blue}\citealt{#1}})}

\newcommand{\introductoryParagraph}[1]{%
\begin{adjustwidth}{1in}{1in}
\textsl{#1}
\end{adjustwidth}
}

\usepackage{gensymb}

\usepackage{fancyhdr}

\usepackage{empheq}
\usepackage{relsize}

\usepackage{upgreek}

% For the diameter symbol
\usepackage{wasysym}

% PACKAGES TO ADD 
%\usepackage{enumitem}   % https://stackoverflow.com/questions/2007627/latex-how-can-i-create-nested-lists-which-look-this-1-1-1-1-1-1-1-2-1-2

\DeclareMathOperator\erf{erf}

%% NOMENCLATURE
% https://www.overleaf.com/learn/latex/nomenclatures
% https://tex.stackexchange.com/questions/112884/how-to-achieve-nomenclature-entries-like-symbol-description-dimension-and-uni
% https://www.youtube.com/watch?v=Ss1XfsaAnfs
% https://tex.stackexchange.com/questions/86666/how-to-create-both-list-of-abbreviations-and-list-of-nomenclature-using-nomencl/87223
\usepackage[intoc]{nomencl}
\makenomenclature

%% this modifies item separation:
\setlength{\nomitemsep}{12pt}

\usepackage{etoolbox}
\renewcommand\nomgroup[1]{%
  \item[\bfseries
  \ifstrequal{#1}{A}{Acronyms}{%
  \ifstrequal{#1}{G}{Greek Symbols}{%
  \ifstrequal{#1}{R}{Roman Symbols}{}}}%
]}
%]\vspace{10pt}} % this is to add vertical space between the groups.

% This will add the units
%----------------------------------------------
\newcommand{\nomunit}[1]{%
\renewcommand{\nomentryend}{\hspace*{\fill}#1}}
%----------------------------------------------