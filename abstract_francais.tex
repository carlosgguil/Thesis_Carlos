
    
\chapter*{Résumé}
    \addcontentsline{toc}{chapter}{Résumé}

Au cours de ces dernières années, l’industrie aéronautique a développé des systèmes de combustion à faibles émissions pour lutter contre le changement climatique. Afin d'atteindre cet objectif, les constructeurs de moteurs aéronautiques ont développé des concepts ciblant la combustion pauvre pour réduire les émissions polluants, notamment les oxydes d’azote (NOx) et le monoxyde de carbone (CO). Ce régime de mélange pauvre peut être obtenu par un placement adéquat de la phase liquide dans la chambre de combustion. Dans cette optique, de nouvelles technologies d’injection de carburant sont apparues, comme les brûleurs à plusieurs étages (MSFI). L’objectif de cette thèse est de développer une nouvelle méthodologie numérique pour la prescription réaliste des sprays modélisés en tant que particules Lagrangiennes, afin de simuler la phase liquide dispersée dans des systèmes MSFI.

Dans un premier temps, les modèles d’injection Lagrangienne sont développés et validés avec une configuration académique de type jet-in-crossflow (JICF). L’aspect théorique de ces nouveaux modèles d’injection, appelés Smart Lagrangian Injectors (SLI), est également détaillé. Les modèles SLI apprennent des données du spray issues de simulations résolues de l’interface liquide-gaz, et utilisent ensuite ces données afin de générer des conditions aux limites concernant les gouttes, qui permettent enfin d’effectuer des simulations Lagrangiennes de la phase liquide dispersée. Par ailleurs, lors de ces simulations de phase dispersée, l’atomisation secondaire des gouttes et l’échange de quantité de mouvement entre le phase dense liquide et le gaz sont également modélisés. Les résultats des simulations d’interface résolue montrent que le comportement physique et la topologie du JICF sont correctement reproduits. Le spray résolu issu de ces simulations est ensuite post-traité pour générer les conditions aux limites de gouttes Lagrangiennes des simulations de phase dispersée. Le spray Lagrangien ainsi produit est comparé à données expérimentales, montrant un comportement physique global satisfaisant et une estimation correcte des débits liquides, mais une sous-estimation de la taille des gouttes causée par le modèle d’atomisation secondaire.

Enfin, la méthodologie SLI est appliquée à l’étage d’injection multipoint du brûleur BIMER, testé au laboratoire EM2C, et plus représentatif des brûleurs industriels. Les modèles SLI sont établis à partir de simulations résolues de l’interface liquide-gaz pour un seul point d’injection, puis appliqués aux simulations de phase dispersée de l’étage multipoint complet, composé de 10 points d’injection. Ces calculs Lagrangiens sont en accord avec les résultats expérimentaux, démontrant la capacité du modèle SLI à générer efficacement des conditions aux limites réalistes pour l’injection liquide afin d’effectuer des simulations liquides dispersées dans des brûleurs MSFI.
