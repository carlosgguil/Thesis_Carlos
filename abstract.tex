\chapter*{Abstract}
    \addcontentsline{toc}{chapter}{Abstract}

{\large \textbf{Modelling of Multipoint Fuel Injection for the Large Eddy Simulation of Aeronautical Combustion Chambers }} \\
    
In the last decades, the aeronautical industry has focused on developing low emission combustion systems to fight climate change. With this objective in mind, aircraft engine manufacturers have developed concepts aimed at burning in lean regimes for reducing pollutant emissions, such as nitrogen oxides (NOx) and carbon monoxide (CO). Lean combustion can be achieved through a proper placement of the liquid phase in the combustion chamber. For this purpose new injection concepts have arisen, such as multi-staged fuel injection (MSFI) systems. The objective of this thesis is the development of a new lagrangian injection methodology for performing dispersed-phase simulations with a realistic prescription of the liquid phase in MSFI systems.

In first place, the lagrangian injection models are developed and validated in an academic kerosene jet in crossflow (JICF) configuration. The theory aspect of the models, named Smart Lagrangian Injectors (SLI), is detailed. SLIs are able to learn spray data from simulations solving for the liquid-gas interface (resolved atomization simulations), and then use these data to prescribe liquid boundary conditions in dispersed-phase computations which model the liquid phase as Lagrangian particles. Furthermore, secondary atomization and momentum exchange between the liquid dense core and the gas are also modeled in the latter.  Results from the resolved atomization simulations show that the JICF physical behaviour and topology can be properly retrieved. The spray resolved from these computations is then post-processed to generate the SLI for prescribing liquid boundary conditions in dispersed phase computations. The resulting spray is validated with experimental data, showing a good physical spray behaviour and a correct estimation of fluxes, but an underestimation in the droplets sizes caused by secondary breakup. 

Finally, the SLI strategy is applied to the multipoint stage of the BIMER multi-staged combustor, tested at EM2C laboratory, which is more representative of industrial burners. Resolved atomization simulations are performed on one single liquid nozzle. SLIs are built from these simulations and applied to the full multipoint stage, consisting of 10 liquid nozzles, for performing dispersed-phase computations of the burner. These computations show a good agreement with experiments, proving the capability of SLI to prescribe realistic liquid boundary conditions for performing dispersed-phase simulations in MSFI burners.

    
\newpage
\shipout\null 
\stepcounter{page}
\newpage
    
\chapter*{Résumé}
    \addcontentsline{toc}{chapter}{Résumé}

{\large \textbf{Modélisation de l'injection multipoint de carburant pour la simulation aux grandes échelles de chambres de combustion aéronautiques }} \\

Au cours de ces dernières années, l’industrie aéronautique a développé des systèmes de combustion à faibles émissions pour lutter contre le changement climatique. Afin d'atteindre cet objectif, les constructeurs de moteurs aéronautiques ont développé des concepts ciblant la combustion pauvre pour réduire les émissions polluants, notamment les oxydes d’azote (NOx) et le monoxyde de carbone (CO). Ce régime de mélange pauvre peut être obtenu par un placement adéquat de la phase liquide dans la chambre de combustion. Dans cette optique, de nouvelles technologies d’injection de carburant sont apparues, comme les brûleurs à plusieurs étages (MSFI). L’objectif de cette thèse est de développer une nouvelle méthodologie numérique pour la prescription réaliste des sprays modélisés en tant que particules Lagrangiennes, afin de simuler la phase liquide dispersée dans des systèmes MSFI.

Dans un premier temps, les modèles d’injection Lagrangienne sont développés et validés avec une configuration académique de type jet-in-crossflow (JICF). L’aspect théorique de ces nouveaux modèles d’injection, appelés Smart Lagrangian Injectors (SLI), est également détaillé. Les modèles SLI apprennent des données du spray issues de simulations résolues de l’interface liquide-gaz, et utilisent ensuite ces données afin de générer des conditions aux limites concernant les gouttes, qui permettent enfin d’effectuer des simulations Lagrangiennes de la phase liquide dispersée. Par ailleurs, lors de ces simulations de phase dispersée, l’atomisation secondaire des gouttes et l’échange de quantité de mouvement entre le phase dense liquide et le gaz sont également modélisés. Les résultats des simulations d’interface résolue montrent que le comportement physique et la topologie du JICF sont correctement reproduits. Le spray résolu issu de ces simulations est ensuite post-traité pour générer les conditions aux limites de gouttes Lagrangiennes des simulations de phase dispersée. Le spray Lagrangien ainsi produit est comparé à données expérimentales, montrant un comportement physique global satisfaisant et une estimation correcte des débits liquides, mais une sous-estimation de la taille des gouttes causée par le modèle d’atomisation secondaire.

Enfin, la méthodologie SLI est appliquée à l’étage d’injection multipoint du brûleur BIMER, testé au laboratoire EM2C, et plus représentatif des brûleurs industriels. Les modèles SLI sont établis à partir de simulations résolues de l’interface liquide-gaz pour un seul point d’injection, puis appliqués aux simulations de phase dispersée de l’étage multipoint complet, composé de 10 points d’injection. Ces calculs Lagrangiens sont en accord avec les résultats expérimentaux, démontrant la capacité du modèle SLI à générer efficacement des conditions aux limites réalistes pour l’injection liquide afin d’effectuer des simulations liquides dispersées dans des brûleurs MSFI.
