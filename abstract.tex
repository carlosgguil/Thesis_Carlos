\chapter*{Abstract}
    \addcontentsline{toc}{chapter}{Abstract}
    
\hl{In the last decades, the aeronautical industry has focused on working towards greener combustion systems to deal with climate change. With the objective of reducing pollutant emissions such as NOx and CO, aircraft engine manufacturers have worked towards combustion chambers which burn fuel at lean regimes.} Lean regimes can be achieved through a proper placement of the liquid phase in the combustion chamber. For this purpose new injection concepts, such as multi-staged fuel injection (MSFI) systems, have arisen. In this concept, fuel injection is divided into two stages: a pilot and a multipoint stage. The objective of this thesis is the development of a new methodology to build lagrangian injectors for performing dispersed-phase simulations with a realistic prescription of the liquid phase in MSFI systems.

In first place, the lagrangian injection models are developed with an academic kerosene jet in crossflow (JICF) configuration. The theory and base of the models, named Smart Lagrangian Injectors (SLI), is detailed. SLIs are able to learn spray data from simulations solving for the liquid-gas interface (resolved atomization simulations), and use these data to prescribe liquid boundary conditions in lagrangian simulations. Furthermore, secondary atomization and the momentum exchange between the liquid dense core and the gas are also modeled in the latter.  Results from the resolved atomization simulations show that the JICF physical behaviour can be properly retrieved. The spray generated is then processed to generate the spray structures conforming the SLI. Then, dispersed-phase computations are performed with the injectors developed. It is shown that the liquid phase can be properly initialized with SLI. The resulting spray is validated with experimental data, showing a good physical spray behaviour but an underestimation in the droplets sizes caused by the secondary atomization model. %Mispredictions of the relative liquid-gas velocities due to the momentum exchange modelling is thought to cause such an aggressive secondary breakup.

Finally, the SLI strategy is applied to the take-off stage of the BIMER multipoint combustor bench tested at EM2C laboratory, which is more representative of industrial burners. While consisting of 10 liquid injectors, resolved atomization simulations have been performed only on one injector due to symmetry of the burner. SLIs are built from these simulations and applied to the full take-off stage for performing dispersed-phase computations of the full burner. These computations show a good agreement with experiments, proving the capability of SLI to prescribe boundary conditions for the liquid phase in MSFI burners.

%Results are then validated with experiments, showing a good spray behaviour and a correct match of droplets sizes.  \hl{The lack of secondary atomization in this configuration is the cause of such accurate prediction in sizes.}





%\hl{In the last decades, the aeronautical industry has focused on working towards greener combustion systems to deal with climate change. With the objective of reducing pollutant emissions such as NOx and CO, aircraft engine manufacturers have worked towards combustion chambers which burn fuel at lean regimes.} Lean combustion can be achieved thanks to a proper placement of the liquid phase which can be done through new injection concepts such as multi-staged injection, in which fuel is injected through a pilot and a multipoint stage. The design of such injectors requires a deep knowledge of the spray produced by the multipoint stage and its effect on the flame. For studying the effect of liquid fuel on combustion, it is paramount to perform an accurate and realistic injection of the spray in reactive simulations. This PhD has aimed at developing a new injection methodology to prescribe spray boundary conditions in disperse phase simulations which are representative of the multipoint injection. Firstly, resolved simulations of primary atomization are performed with a level-set formalisms. The spray resulting from these simulations is used to generate a database which is then processed by the models developed, named Smart Lagrangian Injectors (SLI), to impose realistic spray boundary conditions in dispersed phase simulations where liquid is represented through a lagrangian formalism. In a first step, the models are built and validated with an academic liquid jet-in-crossflow (JICF), which is a representative case of multipoint injection in aeronautical burners. In a second step, the models are then applied to an academic multipoint burner more representative of industrial burners (BIMER).
        
%During the last years, aircraft engine manufacturers have placed reduction of pollutant emissions, such as NOx, and CO as the main consideration for the development of new combustion chambers. With this objective in mind, lean combustion concepts appeared as an alternative for pollutant reduction. It is known 
    
\newpage
\shipout\null 
\newpage
    
\chapter*{Résumé}
    \addcontentsline{toc}{chapter}{Résumé}