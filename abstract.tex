\chapter*{Abstract}
    \addcontentsline{toc}{chapter}{Abstract}
    
During
    
As climate change has become an urgency in the past years, the aeronautical industry needs to shift towards cleaner ways of flying. \hl{This includes hydrogen ...} \\

In the last decades, climate change has put the aeronautical industry 

\hl{OJO A ESTO}:
        
With the objective of reducing pollutant emissions such as NOx and CO, aircraft engine manufacturers have worked towards combustion chambers which burn fuel at lean regimes. Lean combustion can be achieved thanks to a proper placement of the liquid phase which can be done through new injection concepts such as multi-staged injection, in which fuel is injected through a pilot and a multipoint stage. The design of such injectors requires a deep knowledge of the spray produced by the multipoint stage and its effect on the flame. For studying the effect of liquid fuel on combustion, it is paramount to perform an accurate and realistic injection of the spray in reactive simulations. This PhD has aimed at developing a new injection methodology to prescribe spray boundary conditions in disperse phase simulations which are representative of the multipoint injection. Firstly, resolved simulations of primary atomization are performed with a level-set formalisms. The spray resulting from these simulations is used to generate a database which is then processed by the models developed, named Smart Lagrangian Injectors (SLI), to impose realistic spray boundary conditions in dispersed phase simulations where liquid is represented through a lagrangian formalism. In a first step, the models are built and validated with an academic liquid jet-in-crossflow (JICF), which is a representative case of multipoint injection in aeronautical burners. In a second step, the models are then applied to an academic multipoint burner more representative of industrial burners (BIMER).
        
%During the last years, aircraft engine manufacturers have placed reduction of pollutant emissions, such as NOx, and CO as the main consideration for the development of new combustion chambers. With this objective in mind, lean combustion concepts appeared as an alternative for pollutant reduction. It is known 
    
\newpage
\shipout\null 
\newpage
    
\chapter*{Résumé}
    \addcontentsline{toc}{chapter}{Résumé}