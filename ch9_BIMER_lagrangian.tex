\chapter{Spray learning from resolved atomization simulations of BIMER}
	\label{ch:bimer_test_bench}

\section{Introduction}

\section{Injector definition for resolved atomization hole}

\section{Extrapolation of injectors to rest of multipoint holes}

The 

\subsection{Injectors geometry}

\begin{equation}
\boldsymbol{x}_0 =  \begin{pmatrix} - 38.5 ~\mathrm{mm} \\ r \cos \alpha_0 \\ r \sin \alpha_0 \end{pmatrix}
\end{equation}

\begin{equation}
\boldsymbol{x}_i =  \begin{pmatrix} - 38.5 ~\mathrm{mm} \\ r \cos \alpha_i \\ r \sin \alpha_i \end{pmatrix}
\end{equation}


\subsection{General procedure}

\begin{enumerate}

	\item Obtain parameters for SLI of injector 0 (baseline parameters):
	
	\begin{equation}
	\alpha_0  ~~ ; ~~ \boldsymbol{n}_0 ~~ ; ~~ \theta_0 = 90 - \alpha_0 - atan \left( \frac{n_y}{n_z} \right) ~~ ; 
	\end{equation}

	\item Get parameters for SLI of injector $i$ from baseline:
	
	\begin{equation}
	\alpha_i = \alpha_0 - i \Delta \alpha 
	\end{equation}
	
	\begin{equation}
	\boldsymbol{n}_i = 
	\end{equation}
	
	\begin{equation}
	\theta_1 = \theta_0
	\end{equation}
	

\end{enumerate}

\subsection{Definition of coordinate systems and operations}

The global coordinate system is:

\begin{equation}
\boldsymbol{x} =  \begin{pmatrix} x \\ y \\ z \end{pmatrix}
\end{equation}

The local (crossflow) coordinate system is:

\begin{equation}
\boldsymbol{x}^{cr} = \begin{pmatrix} x^c \\ y^c \\ z^c \end{pmatrix}
\end{equation}

with the following equivalences between local and global systems :

\begin{equation}
\boldsymbol{x}^c = \boldsymbol{n}  ~~~~ ; ~~~~ \boldsymbol{z}^c = \boldsymbol{x}  ~~~~ ; ~~~~ \boldsymbol{y}^c =  \boldsymbol{z}^c \times \boldsymbol{x}^c
\end{equation}

where the rotation matrix being:

\begin{equation}
\boldsymbol{R} = \begin{pmatrix} \boldsymbol{x}^{c^T} \\ \boldsymbol{y}^{c^T} \\ \boldsymbol{x}^{c^T} \end{pmatrix}
\end{equation}

More elegantly expresses:

\begin{equation}
\boldsymbol{R} = \begin{pmatrix} x^c_x & x^c_y & x^c_z \\ y^c_x & y^c_y & y^c_z \\ z^c_x & z^c_y & z^c_z \end{pmatrix}
\end{equation}

%\begin{equation}
%\boldsymbol{x} =  \begin{pmatrix} 1 & 2 & -3 \\ 4 & 0 & 1 \end{pmatrix}
%\end{equation}


Transformation for droplet locations is translation + rotation

\begin{equation}
\boldsymbol{x}^c_\mathrm{dr} = \boldsymbol{R} \left( \boldsymbol{x}_\mathrm{dr} -  \boldsymbol{x}_0 \right)
\end{equation}

For droplet velocities, transformation is only rotation:

\begin{equation}
\boldsymbol{u}^c_\mathrm{dr} = \boldsymbol{R} \boldsymbol{u}_\mathrm{dr}
\end{equation}

Inverse transform then:

\begin{equation}
\boldsymbol{x}_\mathrm{inj} = \boldsymbol{x}_0 + \boldsymbol{R}^{-1} \boldsymbol{x}_\mathrm{inj}^c
\end{equation}

\begin{equation}
\boldsymbol{u}_\mathrm{inj} = \boldsymbol{R}^{-1} \boldsymbol{u}_\mathrm{inj}^c
\end{equation}

\section{Conclusion}
