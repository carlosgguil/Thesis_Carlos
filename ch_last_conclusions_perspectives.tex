\chapter{Conclusions and perspectives}

\section*{Conclusions}

In this dissertation, a novel methodology to model liquid fuel injection in multi-staged systems has been proposed. In particular, the focus has been placed on atomizers type jet-in-crossflow (JICF). The new injection models, baptised as Smart Lagrangian Injectors (SLI), are able to learn spray features from resolved atomization simulations, and process those features for prescribing the liquid phase in dispersed-phase computations. In addition, perturbation effects in the gas phase created by the liquid dense core and secondary breakup of lagrangian particles are also accounted for. Two configurations have been studied: an academic JICF which served as the validation case, and a multi-staged injector named BIMER as the application case, which is a configuration more representative of industrial injection systems. The most significant results extracted from each case are summarized next.


\subsubsection*{Kerosene jet in crossflow}

The models have been developed and validated in an academic, non-reactive kerosene JICF studied experimentally by \citeColor[becker_breakup_2002]. Resolved atomization simulations were performed with YALES2 for two interface mesh resolutions. Firstly, the physical behaviour of the JICF was studied. The resolved simulations could correctly capture the two breakup modes (column and surface breakup) observed experimentally.  The jet penetration was quantified and compared to an experimental correlation, showing two different zones: a near-nozzle region belonging to the liquid dense core, where numerical trajectories follow closely the experimental correlation, and a far-nozzle area where the liquid is found in a dispersed phase and larger deviations are obtained. Secondly, the dense core topology and the spray features were postprocessed to build the SLI model. The former estimates the jet breakup point coordinates, which provide input parameters to the Actuator Line Method (ALM) for dispersed-phase simulations. The latter was analyzed through a lagrangian tracking methodology which samples particles when they traverse planes perpendicular to the crossflow direction. The sampled sprays could then be spatially represented through local maps of droplets sizes, velocities and fluxes. Such spatially distributed sprays conform the liquid injectors to prescribe lagrangian droplets in dispersed-phase simulations. \\

In a second step, lagrangian simulations were performed with SLI. Firstly, initial gaseous fields were obtained with two different methodologies to model the disturbance effect caused by the dense core. The first one, which employed the ALM methodology, captured relevant flow features such as recirculation and deceleration regions, yet it could not retrieve all the characteristics of the resolved flow field. The second method, which consisted on prescribing gaseous statistics extracted from the resolved computations into a reduced computational domain, could better capture the gas flow features. Dispersed-phase computations on a high Weber operating point were then performed for both gaseous initial conditions. The prescribed gaseous inlet yielded droplets sizes closer to the experimental results due to a better estimation of the relative liquid-gas velocities, which affected secondary atomization: thus, this gas initial condition was selected for the following studies. The effect of the secondary atomization model was studied next, with the Gorokhovski's stochastic model showing a better performance over the TAB and ETAB ones. Finally, the different liquid phase parameters were analyzed. The most influential ones were found to be the resolution of resolved simulations and the velocities imposed. Regarding the resolution, the injectors obtained from a finer interface cell size produce a better experimental comparison independently of the injection location chosen (while for a coarser resolution, this location has a strong influence on the resulting spray). Regarding liquid velocities, the best results were obtained when prescribing mean values calculated through volume-weighted averages, and then adding a RMS component estimated from a Gaussian distribution. The resulting optimal SLI configuration can accurately capture experimental features such as the spray boundaries, the maximum flux location and the ballistic behaviour. On the other hand, droplets sizes are underestimated, with an experimental deviation of $37~\%$ for the global SMD. Computations on another operating condition at a lower Weber number reduced these deviations to $20~\%$. These results indicate that secondary atomization is the most influential mechanism when SLI is applied to this configuration, and that size underprediction arises due to a misprediction of relative velocities in the dispersed-phase computations. This hypothesis was verified by deactivating secondary breakup for a distance $\Delta x_\mathrm{atom}$ after droplet injection, allowing them to relax towards the gaseous phase. Results showed that SMD increases linearly with $\Delta x_\mathrm{atom}$, approaching the experimental results, yet these sprays produced counter-ballistic profiles which are unphysical. Further work should be devoted to improve prediction of relative velocities in dispersed-phase simulations. Nevertheless, these computations have shown the capability of SLI to prescribe liquid boundary conditions in dispersed-phase simulations of JICF atomizers, paving the way to a realistic modelling of liquid injection in multi-staged injectors relevant to gas turbine combustion systems.


\subsubsection*{Swirled injector BIMER}

In the last part of this thesis, the SLI strategy was applied to model the take-off injection stage of the BIMER buner, tested experimentally by \citeColor[renaud_high-speed_2015] at EM2C. Resolved atomization simulations were performed with two interface resolutions on a single injection point. The liquid jets captured the two breakup modes present in JICF (column and surface breakup). Vertical trajectories showed good agreement with an applicable experimental correlation in the near-nozzle region. Sprays were also analyzed in planes perpendicular to the crossflow, revealing that droplet size convergence is reached for interface resolution (smallest droplets captured have a size of 30 $\mu$m in both simulations) and for axial distance (constant SMD and invariant histograms are obtained at 2 mm downstream the injection nozzle, indicating complete atomization). These sprays were then spatially discretized to yield SLI for liquid fuel injection in lagrangian computations. \\

Dispersed-phase computations were finally performed for the full multi-staged injector. The pilot stage was simulated with the hollow-cone model FIM-UR. The take-off stage was modelled with the SLI built from the resolved atomization simulations. Even if only one liquid injection point was resolved, the full take-off stage (composed of 10 injectors) could be simulated by applying the same SLI to each injector. Three lagrangian computations were performed: one with ALM to model the gaseous phase perturbations, another one where evaporation physics were accounted, and another one without any of both. It was found that secondary atomization did not act in any computation, since all injected droplets were in equilibrium with the surrounding gas. The ALM simulation showed that the actuator proposed could not retrieve the complex gaseous field from the resolved simulations. Nevertheless, comparison with experimental results showed no difference when ALM was included or not, which is due to the absence of secondary atomization. These two simulations showed good experimental agreement qualitatively on the SMD and axial velocity maps, and quantitatively on the global SMD, with differences slightly lower than 30 $\%$. Adding evaporation physics improved results, reducing the SMD deviation up to 5 $\%$. These simulations are, to the author's knowledge, the first lagrangian computations of a complete multi-staged injector where the full multipoint stage is addressed through a realistic 
injection model. The results presented hence demonstrate the effectiveness of the SLI model to accurately simulate multi-staged systems with reduced computational costs, showing also the possibility to account for multiphysics phenomena such as evaporation.

\clearpage




\section*{Perspectives}

This thesis has contributed to the modelling of liquid injection in multi-staged fuel injectors. Future works should focus on improving the different modelling blocks to make dispersed-phase simulations more accurate, and to test the influence of SLI in reactive computations. Therefore, the following perspectives for future extension of this research are proposed:


\begin{itemize}

	\item Account for more realistic physics in liquid injection parameters.  In particular, it is suggested to address the following aspects:
	
	\begin{itemize}
	
		\item Accounting for droplet non-sphericity through modified drag coefficients for lagrangian particles \citepColor[bagheri_drag_2016]. This would increase the momentum transfer after injection for highly deformed structures (such as those corresponding to ligaments in the resolved computations), hence making a more realistic spray transport prior to secondary atomization. 
		
		\item Prescribing droplets velocities conditioned on their sizes through sectional approaches \citepColor[vie_accounting_2013]. This could lead to a better estimation of liquid-gas relative velocities and, therefore, to a more realistic breakup of lagrangian particles through secondary atomization.
		
		\item Consider transient flow rates for prescription of lagrangian droplets, to better mimic the unsteadiness in the dispersed phase region of liquid jets in crossflow.
	
	\end{itemize}
	
	\item Improve modelling of the dense core disturbance effects. The ALM model could be extended to consider complex geometries and net force laws more representative of the jet dense core. Obtention of ALM parameters could also be improved through the employment of multiparameter optimization methods or machine learning tools, such as neural networks. Furthermore, frequential effects could be included in the models to account for the dense core dynamic behaviour.
	
	\item Apply the SLI methodology to other atomization configurations, such as the airblast spray resulting from liquid-wall interactions or the hollow cone generated by pressure-swirl atomizers. The latter configuration was modelled in BIMER with the FIM-UR methodology, hence SLI could be applied to this configuration and compare with the results reported in Chapter \ref{ch9:BIMER_lagrangian}. Application to other cases might also require modelling tools different than ALM to address the gaseous disturbances created by the liquid dense phase, which are dependent on the atomizer.
	
	\item Construct the SLI models from extended databases. Current works are focusing on an Eulerian-Lagrangian coupling to convert the droplets from resolved simulations into Lagrangian Point Particles (LPP) that are later transported in the same simulation \citepColor[janodet_numerical_2022]. Thus, the SLI methodology could be applied in these coupled simulations to generate liquid injectors from lagrangian particles in locations further downstream the dense core. Other possibility is to feed SLI with experimental data on particles when available.  
	
	\item Study the impact of the injection models in reactive simulations. This thesis has shown the possibility to perform lagrangian simulations with SLI in realistic multi-staged injectors, while also accounting for relevant physical mechanisms such as secondary atomization and evaporation. Therefore, SLI is an adequate tool to further analyze the effect of liquid injection in reactive phenomena such as ignition, flame topology, flame-turbulence interaction, thermoacoustic instabilities and pollutant formation. Since BIMER is a reactive test bench which also presents experimental results on combustion simulations \citepColor[renaud_high-speed_2015], the SLI developed in this thesis could be used as starting point for such studies in this test bench.
	
	
	%SLI offers the possibility to analyze the effect of different liquid boundary conditions (e.g. injection locations, prescription of droplets sizes and velocities through different laws) in the evaporation, mixing and combustion processes. In this regard, SLI is a powerful tool to analyze the effect of liquid injection in reactive phenomena such as ignition, extinction, flame topology, flame-turbulence interaction, thermoacoustic instabilities and pollutant formation. Since BIMER is a reactive test bench which presents extensive experimental results on combustion simulations \citepColor[renaud_high-speed_2015], the SLI developed in this thesis for this configuration could be used as starting point for these studies.
	



\end{itemize}

