\chapter{Conclusions and perspectives}

\section*{Conclusions}

In this dissertation, a novel methodology to model liquid fuel injection in multipoint systems has been proposed. In particular, the focus has been set on atomizers type jet-in-crossflow (JICF). The new injection models, baptised as Smart Lagrangian Injectors (SLI), are able to learn the spray from resolved atomization simulations, and process it to prescribe the liquid phase in dispersed-phase simulations. In a first step, the theory of the models has been developed and validated in an academic, non-reactive kerosene JICF studied experimentally by \citeColor[becker_breakup_2002]. In a second step, the SLI strategy has been applied to model the take-off stage of the multipoint injector BIMER, tested by \citeColor[renaud_high-speed_2015] at EM2C, which is a configuration more representative of industrial injection systems. The main conclusions extracted from each case are summarized next.


\subsubsection*{Kerosene jet in crossflow}

\textbf{Resolved atomization simulations} were performed with two mesh resolutions at the interface. Firstly, the \textbf{general physics} of the JICF were studied. The resolved simulations could correctly capture the breakup modes (column and surface breakup) observed experimentally for the operating conditions studied. Surprisingly, the finer resolution could capture the jet instabilities at the windward surface while the coarser one did not. It is thought that this is caused by better resolving the gas phase near the interface for the fine, {although this cannot be stated with clarity}. The penetration of the jet was quantified and compared to experimental correlation, revealing the existence of two zones: a near-nozzle region where numerical trajectories matched the experimental correlation, and a far-nozzle area where they deviated further. The first region coincides with the coherent jet, where the dense core is located, \hl{while the second one corresponds to the dispersed region where a large popoulation of droplets are generated}.


Dispersed-phase simulations for this configuration were then performed with the SLI built from the resolved simulations. 

\subsubsection*{Swirled injector BIMER}

Resolv



\section*{Perspectives}

\hl{Since this work has developed new models}

\begin{itemize}

	\item Propose a different geometry for SLI (or different sampling methodology, see Renaud's comment also).
	
	\item Something related to reactive.
	
	\item Quizas proponer de estudiar cosas sobre la fisica?

\end{itemize}
