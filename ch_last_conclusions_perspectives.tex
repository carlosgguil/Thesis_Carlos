\chapter{Conclusions and perspectives}

\section*{Conclusions}

In this dissertation, a novel methodology to model liquid fuel injection in multipoint systems has been proposed. In particular, the focus has been set on atomizers type jet-in-crossflow (JICF). The new injection models, baptised as Smart Lagrangian Injectors (SLI), are able to learn the spray from resolved atomization simulations, and process it to prescribe the liquid phase in dispersed-phase simulations. In a first step, the theory of the models has been developed and validated in an academic, non-reactive kerosene JICF studied experimentally by \citeColor[becker_breakup_2002]. In a second step, the SLI strategy has been applied to model the take-off stage of the multipoint injector BIMER, tested by \citeColor[renaud_high-speed_2015] at EM2C, which is a configuration more representative of industrial injection systems. The main conclusions extracted from each case are summarized next.


\subsubsection*{Kerosene jet in crossflow}

Resolved atomization simulations were performed with two mesh resolutions at the interface. Firstly, the general physics of the JICF were studied. The resolved simulations could correctly capture the breakup modes (column and surface breakup) observed experimentally for the operating conditions studied. It was found that the finer resolution could capture the jet instabilities at the windward surface while the coarser one did not. It is thought that this is caused by better resolving the gas phase near the interface for the fine, {although this cannot be stated with clarity}. The penetration of the jet was quantified and compared to experimental correlation, revealing the existence of two zones: a near-nozzle region where numerical trajectories matched the experimental correlation, and a far-nozzle area where they deviated further. The first region coincides with the coherent jet, where the dense core is located, while the second one corresponds to the dispersed region where droplets are transported. An analysis of liquid flow rates computed directly from the levelset function showed that fluxes are reduced as the distance from the nozzle increases. It was found that these mass losses were due to small droplets reaching sizes of the order of the mesh resolution, being then removed from the simulation due to the inability of the mesh to transport them further. Secondly, \hl{data necessary to build SLI was obtained from these simulations}. A postprocessing methodology to estimate the dense core breakup coordinates was proposed, which provides input parameters to  later apply the Actuator Line Method (ALM) in dispersed-phase simulation. The resulting spray was finally analyzed in planes perpendicular to the crossflow direction. Droplets were tracked through their center of mass and sampled according to a lagrangian tracking procedure. This methodology could retrieve liquid fluxes similar to the resolved ones. Statistics were obtained for droplets sizes and velocities, which could then be spatially discretized and represented through maps. These local statistics, which showed a good physical behaviour, conform the injectors to prescribe lagrangian droplets in dispersed-phase simulations. \\

Lagrangian simulations for the same configuration were then performed with the SLI built from the resolved simulations. In first place, initial gaseous fields were obtained with two different methodologies to model the disturbance effect of the dense core. The first one, which modelled the dense core through few geometric parameters with 
ALM, was able to capture relevant flow features such as recirculation and deceleration regions, yet it could not retrieve all the characteristics of the flow field. The second methodology, which consisted on prescribing statistics from the gaseous field extracted from the resolved computations in a reduced computational domain, could better capture the overall flow features. Dispersed-phase computations on a high Weber operating point were then performed with SLI for both gaseous methodologies: the prescribed gaseous inlet yielded a better experimental comparison for droplets diameters, probably due to a better estimation of the relative liquid-gas velocities which affected secondary atomization, thus this methodology was selected. The effect of the secondary atomization was studied next, showing that the model chosen can have a strong influence of the spray. The TAB model produced the smallest droplets, followed by the ETAB and the Gorokhovski with tuned model constants, thus the latter one was retrieved. Finally, the different liquid phase parameters were analyzed. The most influential ones were found to be the resolution of resolved simulations and the velocities imposes. A finer resolution yields more converged injectors which, when injected in lagrangian simulations, produce a better experimental comparison independently of the injection location chosen (while for a coarser resolution, this location has a strong influence on the resulting spray). The velocities were found to affect the dispersed phase through two contributions: the estimation of the mean velocity to impose (which produced better results in terms of spray boundaries and ballistic behaviour for a volume-weighted mean than for an arithmetic mean) and the law to prescribe a random component through the RMS velocity (a Gaussian law is recommended). An optimal SLI configuration was selected from all these studies. This one can accurately capture the spray boundaries, the maximum flux location along the $z$ direction and the ballistic behaviour of the spray. On the other hand, the droplets sizes are underestimated in all computations, with a difference of $37~\%$ in the global SMD with respect to experiments. Computations on a lower Weber operating condition showed that the SMD deviations were reduced to $20~\%$. This suggests that the diameter underestimation, which is caused by the secondary atomization, probably arises due to a misprediction of the relative velocities. To assess this hypothesis, an artificial breakup delay was introduced by letting droplets transport for a distance $\Delta x_\mathrm{atom}$ without breaking after injection, allowing them to relax to the gas phase. After this delay, breakup was again triggered. Results show that the size of the resulting droplets increased linearly with the parameter $\Delta x_\mathrm{atom}$, hence the spray SMD approaches the experimental one. On the contrary, the spray showed a counter-ballistic behaviour where the largest droplets were found in the bottom part rather than in the top one. This demonstrates that the size underestimation is caused by an inaccurate prediction of the relative velocities: hence secondary atomization is the most influencing factor in SLI for this configuration. Future efforts on SLI should focus on better retrieving the relative velocities, by addressing the liquid velocities (e.g. through the sectional approach for droplets size-velocities prescription proposed in $\S$\ref{subsec:ch5_learning_SLI}) and/or the gaseous ones (e.g. through improvements in the ALM methodology).


\subsubsection*{Swirled injector BIMER}

fdsa



\section*{Perspectives}

\hl{Since this work has developed new models}

\begin{itemize}

	\item Propose a different geometry for SLI (or different sampling methodology, see Renaud's comment also).
	
	\item Something related to reactive.
	
	\item Extend SLI to other operating conditions.
	
	\item Quizas proponer de estudiar cosas sobre la fisica?

\end{itemize}
