\chapter{Conclusions and perspectives}

\section*{Conclusions}

In this thesis, performed in the context of the ANNULIGhT project,a new methodology to model liquid fuel injection in multipoint systems relevant for aeronautical applications has been developed and tested. The new models, baptised as Smart Lagrangian Injectors (SLI), are able to learn the spray from resolved simulations of the atomization process, where the liquid phase and atomization dynamics are accurately solved, and process it to created boundary conditions which prescribe a realistic, polydisperse spray in dispersed-phase simulations where the liquid phase is represented by lagrangian particles. 

In a first step, a literature review on the injection physical phenomena relevant to multi-staged fuel injection (MSFI) burners and numerical approaches to model injection systems has been performed (Chapter \ref{ch1:introduction}). Among the three injection phenomena existing in MSFI configurations, the liquid jet in crossflow (LJICF) has been chosen for study due to its relevance in MSFI, as they are injected through the multipoint stage which is characteristic of these burners. The review on numerical approaches has been divided into two: one chapter dealing with the methods to simulate resolved atomization, and another chapter on the methods to simulate a dispersed-phase spray. Among the methods dealing with resolved atomization (Chapter \ref{ch2:numerical_methods_resolved_atomization}), the Automatic Conservative Level-Set (ACLS) combined with a Ghost-Fluid Method (GFM) and an Adaptive Mesh Refinement (AMR) strategy to solved dynamically the liquid interface while keeping a reasonable computational cost has been selected. This methodology has been used for the resolved atomization simulations from this thesis with the software YALES2. The review on dispersed-phase methods (Chapter \ref{ch3:disperse_phase_methods}) has presented several approaches to mathematically represent a dispersed spray, among which the lagrangian-point particle representation has been chosen due to its low computational cost, simplicity in numerical implementation, ability to couple with multiphysics phenomena and its extensive usage in the two-phase community for sprays modelling. Then, the state of the art in lagrangian injection models has been summarized in a Venn diagram where previous research work are classified according to the modelling strategies used. The SLI strategy developed in this thesis has been included in this diagram: it combines the learning of a reference spray with inclusion of the secondary breakup of lagrangian particles and modelling the effect of the dense spray structures in the gaseous phase, which is not accounted for in all existing models. A review on the existing methods to model lagrangian sprays in multipoint systems, with special emphasis in previous works on LJICF, shows that currently there are no models combining all three characteristics of SLI (reference spray learning without exclusively relying on empirical laws or results, secondary breakup and dense liquid perturbation effects towards the gaseuos phase), justifying the research gap that this thesis aims at filling.

Part 2 has addressed the SLI models (Chapter \ref{ch4:sli_development}), their construction from resolved atomization simulations of a classical liquid JICF test bench (Chapter \ref{ch5:jicf_resolved_simulations}) and their application to dispersed-phase simulations in the same configuration (Chapter \ref{ch6:jicf_lgs_simulations}). SLI fundamentals are introduced and discussed in Chapter \ref{ch4:sli_development}. A general spray formulation has been used to derive  a mathematical description of the spray, and has been applied to a liquid JICF for explaining how to obtain the statistical magnitudes characterizing the spray. The full computational procedure followed by the SLI is described through a flowchart: the models retrieve the spray data sampled at planes perpendicular to the crossflow from resolved atomization simulations, then process it through a learning procedure in order to produce a lagrangian injector for prescribing the liquid phase in dispersed-phase simulations. Apart from liquid injection, models also account for the dense core blockage effect, which creates perturbations in the gaseous field that are present in the resolved computation but not in the lagrangian one. This blockage is modelled through the Actuator Line Method (ALM) model, in which discrete body forces are specified along a line to mimic the disturbance effects due to the presence of objects. In this work, a simple model based on three parameters representing a liquid dense core is proposed. Then, secondary atomization of droplets is also accounted with three separate, different models: TAB, ETAB and Gorokhovski. The full methodology has been applied to a liquid JICF, for which resolved computations in order to generate a database for building the injectors are reported in Chapter \ref{ch5:jicf_resolved_simulations}. The configuration simulated is the one studied experimentally by \citeColor[becker_breakup_2002], which consists of a high-pressure kerosene JICF representative of gas turbines conditions. For this case, two operating conditions (low and high Weber number at same momentum ratio $q$) and two levelset interface cell sizes (coarse and fine) are simulated. The results show that fine simulations retrieve the windward stabilities of the liquid jet emanating right downstream the nozzle exit, while the coarse one cannot: this greatly affects the atomization and the vertical trajectory of the jet. The trajectories follow accurately the experimental correlation in the near-field of the liquid nozzle, but then the coarse and fine simulations underestimate and overestimate it respectively. The perturbation effect of the jet and the dynamics of the dense core were also studied with the objective of feeding the ALM model in dispersed-phase simulations. Resolved liquid fluxes in planes perpendicular to the crossflow were also quantified, revealing that there is a mass decrease as liquid is injected due to small droplets reaching the mesh resolution:  this can be limited through the optimization of levelset numerical parameters, yet not fully mitigated. Then, the acculumated droplets in planes perpendicular to the crossflow  was globally characterized, and then discretized to provide a spatial repartition of the spray. These discrete sprays showed good behaviours characteristic of typical JICF configurations, and conform the SLI used for prescription of lagrangian droplets in dispersed-phase simulations. These dispersed-phase computations are then reported in Chapter \ref{ch6:jicf_lgs_simulations}. Firstly, a review of the experimental results and past numerical works in the same configuration is done. Sources of uncertainty in PDA measurements are also briefly discussed, with the objective of estimating an order for the errors in the experimental SMD measurements which are not reported in the original work. Then, gaseous boundary conditions for initializing the simulations are obtained with two methods: the ALM methodology and a prescribed gaseons inlet strategy to better match the perturbed field. The dispersed-phase computations are then performed, for which a parametric analysis of all relevant parameters is performed for the high Weber condition. This analysis is divided into three parts: effect of gaseous boundary conditions, effect of the secondary atomization model, and effect of liquid boundary conditions. Studying the effect of the gaseous field shows that the prescribed inlet methodology provides a better lagrangian field than ALM, and shows a high sensitivity of the lagrangian magnitudes with the gas field. This is attributed to the fact that an inaccurate modelling of the gaseous field creates high deviations in the relative velocities, affecting hence secondary atomization and creating very small lagrangian particles. Indeed, the prescribed inlet also predicts smaller lagrangian particles, but closer to the experimental values than for the ALM simulations, and overall the sprays show a good behaviour in terms of ballistics and flux distribution. It is then concluded that the simple ALM model proposed cannot properly represent the gaseous phase, and hence the prescribed inlet methodology is chosen for future studies. Regarding secondary breakup, the choice of the model to employ seems to be determinant: the TAB model greatly underestimates the particles sizes, while the ETAB improves its behaviour. However, the Gorokhovski model with tuned constants provides the most accurate particles sizes, and hence is taken for the final study of the liquid boundary conditions. For this last study, the most sensitive parameters found were the level-set resolution, the average velocity prescribed and the law for specifying a RMS components in the velocities: it is recommended then to use respectively finer injectors for the resolution, volume-weighted average velocities and a gaussian law for RMS dispersion. Overall, all tests performed showed, in the best of the cases, an underestimation of global SMD of 37 $\%$. This value got reduced to 20 $\%$ for the low Weber operating condition, which could indicate that SLI could work better at cases working at lower gaseous Reynolds number due to a less aggressive secondary atomization or to a better resolution of the particles in the levelset simulations. In any case, underestimation of particles sizes is thought to be due to a mismatch in the relative velocities which produces a strong breakup through the breakup models. To assess this, an atomization delay was tested to let droplets exchange momentum with the gaseous phase before being broken, showing that actually the final sizes obtained were larger at the detriment of getting a counter-ballistic sprays, which is unphysical. Overall, despite the lower particles sizes obtained, the SLIs methodology has demonstrated to produce a lagrangian spray with a correct JICF behaviour in a high-pressure configuration.

\clearpage

Finally, part 3 has applied the SLI method to an academic multipoint burner representative of industrial injectors named BIMER and tested at EM2C \citepColor[renaud_high-speed_2015]. Chapter \ref{ch7:bimer_test_bench_and_aero} has made a first set of gaseous simulations with two operating points: one with available experimental data on aerodynamic field for validating these simulations, and another one with available experimental data on the spray field for latter validating the lagrangian simulations. The full multipoint burner is geometrically complex and consists of 10 injection holes in the take-off stage. However, due to rotational symmetry in this stage, only one multipoint hole needed be solved and processed with the SLI methodology. The resolved simulations from this case are shown in Chapter \ref{ch8:bimer_resolved_atomization}. Experimental data was not available for this stage to validate these simulations: nevertheless, a comparison with an experimental trajectory from other works valid for the operating condition simulated showed good agreement in the near-field trajectory. The same analyses as the ones in the resolved simulations of Chapter \ref{ch5:jicf_resolved_simulations} were performed, eventually succeeding in retrieving the spray in planes perpendicular to the crossflow direction (which in a swirled injector is not straightforward to define) and discretizing it for for producing SLIs. These injectores were then used in Chapter \ref{ch9:BIMER_lagrangian} for dispersed phase computations. In first plane, the perturbation created by a ALM model was verified, showing that in this case ALM cannot properly retrieve the gaseous field. However, this was later found not to affect the spray flow field, since secondary atomization did not take place in any simulation performed. For the liquid boundary conditions, the multipoint stage was initiated with SLIs prescribed at the vinicity of each injection hole, while the LISA model was used for the pilot stage. A total of three simulations were performed, in which it was shown that ALM has no significant effects in the lagrangian sprays and that evaporation plays a fundamental role. In general, all sprays displayed a good experimental comparison for the SMD and mean axial velocity maps. Errors with respect to experiments for the global SMD were of the order of 30 $\%$ for simulations without evaporation, while this value was reduced to 5 $\%$ when evaporation was taken into account. This shows the importance of including heat and mass transfer in this configuration, since the ambient gas is preheated at a relatively high temperature and such mechanisms should not be neglected. These unprecedented simulations demonstrate indeed that SLI is a promising methodology to prescribe realistic liquid boundary conditions in multipoint injectors.

\section*{Perspectives}

This work has developed from scratch a new injection methodology. This one is based on performing dispersed-phase computations with realistic spray boundary conditions learnt from a resolved atomization simulation database. After all the analyses performed in this work, the following areas leave room for possible future improvements:

\begin{itemize}

	\item The SLI models have defined a convergence criterion based on a NSME which depends only on the droplets sizes. When performing convergence-driven discretization, the results were found not to be as good as expected. As so, the NSME definition could be extended to include other spray variables, such as velocities. The convergence-driven discretization procedure could also be double-checked.
	
	\item ALM was found not to properly retrieve the gaseous perturbations in any case. This is thought to be due to the simple model proposed and to the way in which its variables were obtained (in-hand tuning). As so, it is proposed to extend the ALM model to account for more parameters and to define these ones through a multiparameter optimization procedure. Furthermore, inclusion of spectral effects accounting for the dynamics of the dense core, such as the breakup frequencies reported in Table \ref{tab:jicf_ligament_shedding_fs_and_tau_str}, would also improve the reliability of the model.
	
	\item In terms of spray injection, a more accurate prescription of boundary conditions could be performed by injecting droplets velocities conditioned by their injected sizes following the sectional approached briefly discussed in $\S$\ref{subsec:SPS_inhomogeneity_sprays}.
	
	\item Dispersed-phase simulations could add more modelling tools to account for more realistic physics such as the deformation of droplets injected through a variable drag coefficient \citepColor[bagheri_drag_2016], subgrid models for dispersion of gaseous velocities, and coalescence.
	
	\item The SLI models could be applied to other operating conditions (such as lower Reynolds number ones) and to other injector configurations (such as hollow cone) as directly detailed in this work. It would be interesting to see this methodology in other applications, since as shown in this work the results improve in terms of particles sizes when changing the operating condition from high to low Reynolds ($\S$\ref{subsec:JICF_LGS_params_OP}) and when applying the SLI to an academic multipoint injector of complex geometry (Chapter \ref{ch9:BIMER_lagrangian}).


\end{itemize} 

%\begin{itemize}
%
%\item Chapter 4: SLI
%
%	\begin{itemize}
%
%		\item Improve the convergence criterion by adding other magnitudes to the NMSE definition: velocities, for example.
%		
%		\item Apply the SLI methodology to other configurations, such as hollow cone or airblast.
%		
%		\item ALM: include spectral information (time-varying force application).
%
%	\end{itemize}
%
%\end{itemize}


%
%
%Chapter \ref{ch6:jicf_lgs_simulations} ...

%Part 2 has addressed the SLI models, their construction from resolved atomization simulations of a classical LJICF test bench and their application to dispersed-phase simulations in the same configuration. SLI fundamentals are introduced and discussed in Chapter \ref{ch4:sli_development}. A general spray formulation has been used to derive  a mathematical description of the spray, and has been applied to a liquid JICF in order to explain how the statistical magnitudes characterizing the spray are obtained. The full computational procedure followed by the SLI is described through a flowchart: the models retrieve the spray data sampled at planes perpendicular to the crossflow from resolved atomization simulations, then process it through a lagrangian injectors learning procedure in order to produce a lagrangian injector for initializing the spray phase in dispersed-phase simulations. The injectors learning process is thoroughly describe, which consists on sampling resolved droplets when these cross planes perpendicular to the crossflow and process their volume, center of mass location and velocities, and eccentricity radii in order to characterize the spray by their SMD, droplets distribution function, fluxes, and mean and RMS velocity components. The spray sampling process is averaged with time, and the spray convergence is checked through a proposed Normalized Mean Square Error (NMSE) criterion which monitors the evolution of the spray size distribution with time. Then, the full sampled spray is discretized in a grid composed of several probes to get an in-plane spatially discretized, polydisperse lagrangian injector. The discretization process can be done in two ways: ad-hoc, meaning that the grid size is given as input, or through a conergence-driven discretization process, in which a quadtree refinement process creates an automatic grid according to the local convergence of the sprays contained within the probes. Apart from the lagrangian injectors, the methodology also includes a dense core learning process through which the perturbation effect of the liquid column to the air, which is not a priori accounted for in the dispersed-phase computations, can be considered through an Actuator Line Model (ALM). The dense core is extracted from the resolved atomization simulations by characterizing its topology (defined by the breakup point coordinates and dense core with at the breakup point) and the force imposed by the dense core (which is calculated by obtaining the pressures in the windward and leeward sides of the jet). Then, the ALM model mimics its perturbation effect by applying discrete forces in the dispersed-phase simulations in a region known as actuator: the forces imposed represent the calculated dense core forces, while the actuator region represents its geometry. Finally, the models account for secondary atomization models to consider further breakup of the lagrangian droplets in case they are not in equilibrium with the surrounding environment. Three models have been implemented so far: the Taylor Analogy Breakup (TAB), the Enhanced TAB (ETAB) which is an improvement of the former one, and the Gorokhovski's stochastic model.
%
%Chapter \ref{ch5:jicf_resolved_simulations} ...
%
%
