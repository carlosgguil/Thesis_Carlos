\chapter{Validation in liquid jet in crossflow}
	\label{ch:jicf_lgs_simulations}


Describe here all our results from the lagrangian simulations:

\begin{itemize}

	\item Effects of applying full workflow: w/wo ALM, w/wo secondary atomization ...
	
	\item Mesh convergence study: specify it
	
	\item Validation with experiments
	
	\item Mass conservation issues: lagrangian tracking, etc. 
	

\end{itemize}

\section{Introduction}

\section{Models sensitivity}

ALSO: effect of one and two-way coupling !! Ver articulos 2010, 2011 Li 

\subsection{Effect of injection conditions}

\subsection{Effect of secondary atomization model}

\subsection{Effect of dense core blockage effect model}

\section{Results}

\subsection{Mesh convergence study}

\subsection{Validation with experiments (quantitative/qualitative)}

\subsection{Trajectories}

In order to illustrate the lagrangian trajectories and the continuity with respect to the resolved jets, a volume fraction field can be defined in the lagrangian simulations:

\begin{equation}
\alpha_l \left( \textbf{x}, t \right) = \frac{V_l \left( \textbf{x}, t \right)}{V_{el}}
\end{equation}

where $V_{el}$ is the volume of the element in the eulerian mesh grid. Therefore, the volume fraction is a magnitude defined in the main eulerian grid. Since the dispersed phase is not directly represented in this grid but by lagrangian particles, $V_l \left( \textbf{x}, t \right)$ is calculated by interpolating the volume of the particles located within each element at each iteration.

\section{Conclusions}