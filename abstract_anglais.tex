\chapter*{Abstract}
    \addcontentsline{toc}{chapter}{Abstract}
    
In the last decades, the aeronautical industry has focused on developing low emission combustion systems to fight climate change. With this objective in mind, aircraft engine manufacturers have developed concepts aimed at burning in lean regimes for reducing pollutant emissions, such as nitrogen oxides (NOx) and carbon monoxide (CO). Lean combustion can be achieved through a proper placement of the liquid phase in the combustion chamber. For this purpose new injection concepts have arisen, such as multi-staged fuel injection (MSFI) systems. The objective of this thesis is the development of a new lagrangian injection methodology for performing dispersed-phase simulations with a realistic prescription of the liquid phase in MSFI systems.

In first place, the lagrangian injection models are developed and validated in an academic kerosene jet in crossflow (JICF) configuration. The theory aspect of the models, named Smart Lagrangian Injectors (SLI), is detailed. SLIs are able to learn spray data from simulations solving for the liquid-gas interface (resolved atomization simulations), and then use these data to prescribe liquid boundary conditions in dispersed-phase computations which model the liquid phase as Lagrangian particles. Furthermore, secondary atomization and momentum exchange between the liquid dense core and the gas are also modeled in the latter.  Results from the resolved atomization simulations show that the JICF physical behaviour and topology can be properly retrieved. The spray resolved from these computations is then post-processed to generate the SLI for prescribing liquid boundary conditions in dispersed phase computations. The resulting spray is validated with experimental data, showing a good physical spray behaviour and a correct estimation of fluxes, but an underestimation in the droplets sizes caused by secondary breakup. 

Finally, the SLI strategy is applied to the multipoint stage of the BIMER multi-staged combustor, tested at EM2C laboratory, which is more representative of industrial burners. Resolved atomization simulations are performed on one single liquid nozzle. SLIs are built from these simulations and applied to the full multipoint stage, consisting of 10 liquid nozzles, for performing dispersed-phase computations of the burner. These computations show a good agreement with experiments, proving the capability of SLI to prescribe realistic liquid boundary conditions for performing dispersed-phase simulations in MSFI burners.
