\chapter*{Acknowledgements}
    \addcontentsline{toc}{chapter}{Acknowledgements}
    
Dans \textsl{Siddhartha} \citepColor[hesse_siddhartha_1922], le passeur de la rivière apprend au protagoniste comment traverser celle-ci. D'abord il l'aide à parcourir cette rivière, jusqu'au moment où le protagoniste arrive à la franchir lui même, apprenant à écouter le fleuve et comprenant son esprit. \\  % p. 115
%
Je remercie d'abord mes deux passeurs, Léa Voivenel et Renaud Mercier, qui m'ont guidé pour comprendre et essayer de traverser cette rivière que l'on appelle recherche. Leurs conseils, leur inestimable aide, leur grande disponibilité et leur patience m'ont permis d'aboutir ce projet qui m'a prit tant d'effort et de temps dans les dernières années.

Je dois exprimer ma profonde gratitude à Thierry Poinsot pour avoir dirigé cette thèse et pour m'avoir accueilli au CERFACS pour les derniers mois de ma thèse. Je le remercie profondément d'avoir pris le temps, malgré son calendrier serré, d'assister aux réunions que m'ont permis d'améliorer ce travail.

Je souhaiterais aussi remercier les membres du jury d'avoir assisté à ma soutenance. Merci d'abord à Aymeric Vié et Guillaume Balarac d'avoir accepté d'être  rapporteurs, et d'avoir pris le temps de lire et évaluer ce manuscrit (qui n'est pas le roman le plus amusant du monde...).  Merci a François-Xavier Demoulin d'avoir accepté de présider le jury, et à Jean-Luc Estivalèzes, Marcos Carreres, Vincent Moureau et Stefano Puggelli d'avoir évalué  mon travail lors de la soutenance. Aussi, merci à Stefano et à Vincent pour nos discussions et leurs conseils pendant ma thèse, à Safran ou en visio. 

Maintenant, je voudrais remercier les personnes qui m'ont accompagnés dans cette aventure, en faisant un petit \textsl{road trip} qui va du nord de la France au plus sud, de Paris à l'Espagne. A Paris, je souhaiterais remercier d'abord tous les gens que j'ai trouvés à Safran Tech, dans les pôles "Energie et Propulsion" et "Digital Sciences and Technologies". Merci à Melody Cailler et à Julien Leparoux pour tous nos échanges et d'avoir partagé son expérience sur le diphasique et sur YALES2 avec moi. Je souhaiterais remercier aussi tous les membres du project Marie-Curie ANNULIGhT, avec qui on a partagés des bons moments dans différentes villes européennes (comme Toulouse et Munich), et à qui je souhaite le meilleur pour la suite ! Passant au côté personnel, je ne peux que être reconnaissant d'avoir trouvé des gens comme Romain Janodet, mon copain d'enjeux diphasiques et des hauts et bas de thèse (et avec qui j'espère pouvoir collaborer dans un avenir proche); Gabriel, compagnon de bureau d'abord, puis pote d'aventures à Paris, Lyon et Barcelone (il nous manque Raz toujours!); Quentin, le "bon breton" avec qui avec qui j'ai affronté des moments difficiles; Clement, le \textsl{loco} qui a quitté l'endroit pour aller faire des avions (et qui m'a laissé tout seul les midis dans la salle de sport); Jean-Marie, le master versaillais (car lui, lui il est français) du control et du pilotage; et à Deepali, qui a quitté la region parisienne trop tôt pour faire de l'éolienne dans les champs de tulipes néerlandais (now it's your turn to get that PhD, go kill it!).
  
Toujours dans la capitale, la Ville Lumière n'aurait pas été aussi charmante sans la compagnie de quelques personnes. I would like to thank then to all the people I met during those three-and-something years, that went from the CiteU to Gentilly with some period in the outskirts (long monhts though): Miguel (gran amigo, buen compañero de viajes, mejor coloc), Diego (con quien sigo compartiendo cervezas en Barcelona), Pardis (and her invitations to nice gatherings in Villejuif), Javid (congratulations to you too for your recent defense!), Zhila (whose kindness and welcoming are beyond human nature), Jesús (albaceteño en la gran ciudad, y compañero de \textsl{parades} y de buenos viajes), Federico (argentino loco, buen compañero de fiestas, mejor profe de salsa), Ana (arquitecta \textsl{granaína} ahora más parisina que nadie), Antonio (el catasturiano que nos contagió la moda de las camisas horteras), Gonzalo y Bérangère (que van siempre en pack, ella ya doctora y él quasi-doctor), Laura A. (experta en supervivencia de tesis, y en comprar packs de cerveza para invitarnos a todos), Carmen y Fernando (también en pack, y pronto con alianzas incluidas!), Mar (presente durante el muy principio y el muy final de esta época), Toño (inolvidable compañero de farras, que la providencia ha querido que se sigan dando de vez en cuando) y Jasone (la vasca que más me ha soportado en mi encierro de teletrabajo tras la llegada del pinche virus).

Let's now move souther and stop for a while at \textsl{la ville rose}, Toulouse. Despite the few months spent there (btw very painful ones, as most of the time was spent sitted writing this thesis), their human quality was more than excellent. D'abord, je veux remercier les gens du CERFACS pour leur charmant accueil. En particulier, merci à Michèle pour son amabilité et son aide avec les affaires administratives, et à Gérard, Fabrice et Isabelle du équipe CSG pour leur aide avec les affaires techniques. In the personal plane, I can only be grateful for meeting such fantastic people such as Walter (with whom I have shared adventures since the good times at VKI), Abhijeet (also Annulight partner, the kindest indian heart on the planet), Varun (who made me discover the delices of indian food), Patrick (Glück in dein Doktorarbeit mit dem Rotation-Feuer!), Gregory (now a parisian wandering around Châteaufort), Héctor (dele bien a ese amoniaco, algún día conseguirá su sueño de vivir de rentas wey), Sriram (the expert in covid simulations).

Cruzando los Pirineos hacia el sur, me gustaría empezar por agradecer en Barcelona el apoyo moral de mis colegas del BSC y de mi familia de aquí, fundamental en este periodo de correcciones y lectura de tesis que se ha alargado más de lo previsto. Bajando ahora a Valencia, agradezo a mis compañeros de la universidad su amistad y apoyo, a pesar de vernos un par de veces al año (eso tiene la vida, que cada cual sigue su camino, aunque por fortuna estos se siguen cruzando). Y una pequeña mención (aunque no menos importante) a Elena, Loli y Sara, porque lo que Leeds ha unido que no lo separen los años. 

Para terminar, acabamos este viaje en Requena, donde empezó todo. Un inmenso gracias a mis amigos por su incondicional apoyo y por estar siempre ahí: Juan el maestro del saxofón, Viseras \textsl{Willkommen}, Garban el maestro del basket (y el huevón huevón), Borjeta y nuestras tardes en el perro, Julio y sus \textsl{Amunt Valencia nano}, Pepe y sus ironías mordaces, Perelló el \textsl{quant} que llega siempre tarde a todas partes, Inés y nuestras interminables charlas y confidencias (apoyo incondicional en los momentos más duros, no cambies nunca rubia), y al desaparecido Torres (¡pronto papá Torres!). A todos vosotros, por tantos años juntos, y por los que nos quedan.

Llegando al final, no puedo sino agradecer a mi familia su incondicional apoyo a lo largo de los años. A mi tío Eloy y a mi abuela Nieves, por todo su cariño desde que era pequeño y pasaba los días de verano en el Rebo. A mi hermano Javi, por todas nuestras conversaciones sobre cualquier tema y sus consejos respecto al noble arte de ir al gimnasio. Y a mis padres José y Gloria, por dármelo todo y apoyar todas mis decisiones a pesar de mi empeño por estar fuera de casa (aunque cada vez más cerca).

Infine, le parole non possono esprimere la mia gratitudine per aver trovato Laura in questo viaggio. Il suo amore, il suo sostegno incondizionato e il suo atteggiamento verso la vita mi hanno fatto superare i momenti brutti (che lei, nonostante gli inconvenienti, ha condiviso con me) e godermi al massimo quelli belli. Spero di poterti ripagare negli anni che passeremo insieme, ovunque ci porti la vita. Grazie anche alla famiglia di Laura per la loro gentilezza e ospitalità nella loro casa nella più grande isola italiana.





%\subsection*{Financement}
%
%This PhD thesis has been funded by the European Union Horizon 2020 research and innovation program under the Marie Sklodowska-Curie grant agreement No. 765998 in the project ANNULIGhT. Computer resources provided by GENCI, France, under the allocations A0092B11072 and A0092B10157.

%NOTE: A0092B11072 corresponds to Safran Tech (Mélody), A0092B11072 to Cerfacs one (Florent/Eleonore)
