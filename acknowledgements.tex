\chapter*{Acknowledgements}
    \addcontentsline{toc}{chapter}{Acknowledgements}
    
Dans \textsl{Siddhartha} \citepColor[hesse_siddhartha_1922], le passeur de la rivière apprend au protagoniste comment croisser celle-ci. D'abord il lui aide à croisser et a entendre cette rivière, jusqu'au moment ou Siddhartha arrive à la traspasser lui même, apprendant à écouter le fleuve et comprenant son esprit. \\  % p. 115
%
Je remercie d'abord à mes deux passeurs, Léa Voivenel et Renaud Mercier, qui m'ont  à croisser et à entendre cette rivière qu'on appelle recherche. Ses conseils, son inestimable aide, sa constant disposition et sa patience m'ont permi d'aboutir ce projet qui a prit autant d'effort et de temps dans ma vie.

Je dois exprimer ma profonde gratitude à Thierry Poinsot pour diriger cette thèse et pour m'avoir accueilli au CERFACS pour les dernièrs mois de ma thèse. Je remercie vraiement le temps qu'il a pris, malgré son calendrier serré, pour assister aux réunions que m'ont permis d'ameliorer les modèles et les simulations.

Je souhaiterais aussi remercier aux membres du jury pour assister à ma soutenance. Merci à Aymeric Vié et Guillaume Balarac pour accepter d'ètre rapporteurs et pour prendre le temps de lire et évaluer ce manuscrit (qui n'est pas le roman le plus amusant du monde...).  Merci a François-Xavier Demoulin pour accepter de présider le jury, et à Jean-Luc Estivalèzes, Marcos Carreres, Vincent Moureau et Stefano Puggelli pour évaluer mon travail lors de la soutenance. Aussi, merci à Stefano et à Vincent pour nos discussions et ses conseils, à Safran ou en visio. 

Maintenant, je dois rendre grâce en faisant un petit \textsl{road trip} qui va du nord de France au plus sud, de Paris à l'Espagne. 

A Paris, je souheterais remercier d'abord les gens qui j'ai trouvés à Safran Tech, dans les pôles "Energie et Propulsion" et "Digital Sciences and Technologies". D'abord (et evidemment), merci encore une fois à Léa et à Renaud pour m'acueillir dans ses départements. Du còté technique, merci à Melody Cailler et à Julien Leparoux pour tous nos échanges et pour partager son expérience sur le diphasique et sur YALES2 avec moi.  Et du côté personnel, je ne peux que être reconnaissant d'avoir trouvé à des gens comme Romain Janodet, mon copain d'enjeux diphasiques et des hauts et bas de thêse (et avec qui j'espere pouvoir collaborer dans un avenir proche); Gabriel Mondin, compagne de bureau d'abord, puis pote d'aventures à Paris, Lyon et Barcelone (il nous manque Raz toujours!); Quentin Holka, le bon breton avec qui nous nous sommes appuyés pour sortir de l'abysme dans les moments les plus durs; Clement Pornet, le \textsl{loco} d'ouf qui a quitté l'endroit pour aller faire des avions (et qui m'a laissé tout seul dans la salle de sport); Jean-Marie Kai, le master versaillais (car lui, lui il est français) du control et du pilotage; et à Deepali Singh, qui a quitté la region parisienne trop tôt pour faire de l'eolienne chez les champs de tulipes néerlandais (now it's your turn to get that PhD, go kill it girl!).
  
\textbf{Paris (hors Safran}).  Toujours dans la capitale, la Ville Lumière n'aurait pas été aussi charmant sans la compagnie de quelques personnes.

In first place, I should start with the flashbacs in the CIUP, my first home in Paris.



\textbf{Toulouse}. 

\textbf{Barcelone}. Cruzando los Pirineos hacia el sur, me gustaría empezar por agradecer en Barcleona el apoyo moral a mis colegas del BSC y a mi familia de aquí, fundamental en este periodo de correcciones y lectura de tesis que se ha alargado más de lo previsto. 


\textbf{Valencia (mirlos)? Maybe not}


\textbf{Requena}


\textbf{Family}


\textbf{Laura}


\subsection*{Financement}

This PhD thesis has been funded by the European Union Horizon 2020 research and innovation program under the Marie Sklodowska-Curie grant agreement No. 765998 in the project ANNULIGhT. Computer resources provided by GENCI, France, under the allocations A0092B11072 and A0092B10157.

%NOTE: A0092B11072 corresponds to Safran Tech (Mélody), A0092B11072 to Cerfacs one (Florent/Eleonore)
