\chapter*{Acknowledgements}
    \addcontentsline{toc}{chapter}{Acknowledgements}
    
Dans \textsl{Siddhartha} \citepColor[hesse_siddhartha_1922], le passeur de la rivière apprend au protagoniste comment croisser celle-ci. D'abord il lui aide à croisser et a entendre cette rivière, jusqu'au moment ou Siddhartha arrive à la traspasser lui même, apprendant à écouter le fleuve et comprenant son esprit. \\  % p. 115
%
Je remercie d'abord à mes deux passeurs, Léa Voivenel et Renaud Mercier, qui m'ont appris à croisser et à entendre ces rivières qu'on appele science et recherche - même s'il me reste encore un chemin, peut-être inatteignable, pour bien leur comprendre. Ses inestimables conseils, son aide, sa constant disposition et sa patience (infinie celle-ci) m'ont permi d'aboutir ces travaux de thèse qui ferment un chapitre important dans ma vie. 

Merci à Thierry Poinsot pour diriger cette thèse et pour m'avoir accueilli au CERFACS pour les dernièrs mois de ma thèse.  

Je souhaiterais aussi remercier aux membres du jury de ma thèse. D'abord, merci aux rapporteurs Aymeric Vié et Guillaume Balarac pour prendre le temps de lire et évaluer mon rapport. Merci a François-Xavier Demoulin pour accepter de présider le jury, et aux membres Jean-Luc Estivalèzes, Marcos Carreres, Vincent Moureau et Stefano Puggelli pour assister à ma soutenance et pour ses questions. En particulière, je 

De façon plus géneralé, je vais commencer remercier les gens du nord (Paris) au sud (Toulouse - Barcelone).

Et blabla

\subsection*{Financement}

This PhD thesis has been funded by the European Union Horizon 2020 research and innovation program under the Marie Sklodowska-Curie grant agreement No. 765998 in the project ANNULIGhT. Computer resources provided by GENCI, France, under the allocations A0092B11072 and A0092B10157.

%NOTE: A0092B11072 corresponds to Safran Tech (Mélody), A0092B11072 to Cerfacs one (Florent/Eleonore)