\chapter{Gaseous flow in BIMER multipoint injector}
	\label{ch:bimer_test_bench}

\section{Introduction}

The jet in crossflow atomizer from the previous chapters is an academic test case of fuel representative of LPP systems. Its study is of interest for aeronautical gas turbines since it uses kerosene as injected liquid and it has been tested in a high-pressure environment. Yet, its

The fact that the injected liquid is kerosene is 

Here we describe the aerodynamic simulations performed with BIMER. Two operating points are simulated:

\begin{itemize}

	\item \citeColor[providakis_etude_2013] for \textbf{validation} of aerodynamic simulations, since this operating point shows experimental data for these conditions. Some experimental results are shown in the thesis of \citeColor[cheneau_etude_2019]
	
	\item \citeColor[renaud_high-speed_2015] for \textbf{application} only, once validation has been done. This condition does not present data on the aerodynamic state, but it will be used to simulate the spray (firstly resolved atomization without validation, and later validation on the lagrangian simulations).

\end{itemize}

\section{Experimental setup}

Dodecane properties are shown in Table \ref{tab:dodecane_properties}

\begin{table}[!h]
\centering
\caption{Physical properties of dodecane fuel.}
\begin{tabular}{|c|c|c|}
\hline
$\rho$ [kg m$^{-3}]$   & $\mu$ [Pa s]   & $\sigma$ [N/m]  \\
\hline
750 & & \\
\hline
\end{tabular}
\label{tab:dodecane_properties}
\end{table}


\section{Choice of operating points}

\section{Numerical setup}

\section{Validation of gaseous field}

Show nice streamlines (Fig. 6.4 Esclapez-1) and PVC. It would be nice to compare them for both operating conditions

\section{Application conditions}

\section{Conclusion}

%\section{Validation (Providakis 2013)}
%
%
%\subsection{Quantitative data}
%
%\subsection{Qualitative data}
%
%\section{Application (Renaud 2015)}
%
