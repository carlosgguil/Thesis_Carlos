\chapter{Numerical methods to simulate dispersed phase}
\label{ch3:disperse_phase_methods}

\section{Introduction}

The previous chapter presented numerical methodologies applicable to separate two-phase flows. These are useful for solving problems where the dynamics of atomization need to be accurately resolved. Nonetheless, those methods cannot be applied in the dispersed phase regime, where atomization is complete (or almost complete) and a spray composed of individual droplets is formed. Such problems, often found when studying reactive processes where fuel is injected into a combustion chamber, need a different representation of the liquid phase so that 1) the spray can be transported with acceptable computational costs and 2) can include more complex physics relevant to reactive problems, such as evaporation and combustion.

The spray generated in the dispersed phase regime is mainly distinguished from liquid in the separate regime by the following features (represented in Figure \ref{fig:atomization_regimes_herrmann}):

\begin{itemize}

	\item The characteristic length scales of the particles. In dispersed phase flows, these ones are low and usually smaller than the resolution of the main grid.
	
	\item The value of the liquid volume fraction $\alpha_l$. In dispersed phase flows, $\alpha_l < 1$. According its value, one can distinguish between dense regime (moderate values of $\alpha_l$, particles are close to each other) and dilute regime (lower values of $\alpha_l$, usually below $10^{-3}$, particles are far from each other).

\end{itemize}

The numerical formalisms to resolved dispersed phase flows will hence depend on these two characteristics. It is also important to consider the interaction with the gaseous phase, since this one will depend by the resolution of the main grid. The interaction between the liquid and gaseous phases in dispersed phase flows can be quantified by means of the Stokes number $St$, defined as the ratio between the characteristic time-scale of a liquid particle $\tau_p$ and a characteristic time of the gaseous phase $\tau_g$: 

\begin{equation}
\label{eq:Stokes_number_definition_general}
St = \frac{\tau_p}{\tau_g}
\end{equation}

A classification of numerical methods to simulate dispersed phase flows based on the volume fraction and the Stokes number has been done by \citeColor[balachandar_scaling_2009], shown in Figure \ref{fig:balachandar_numerical_methods_representation}. As it can be seen, the Stokes number will depend on the numerical methodology used to resolve the gaseous phase: in DNS, the smallest scales of turbulence with characteristic size $\upeta$ will be resolved and will have a characteristic time $\tau_k$, while in LES the smallest scales resolved $\upxi$ will be larger and their time-scales will be different ($\tau_\upxi$). Regarding the volume fraction, coupling strategies between liquid particles and gas can be used depending on its value: in the dilute regime particles are far from each other and the interactions among them can be neglected (one and two-way coupling), while in the dense regime the interaction between particles must be taken into account (four-way coupling). The difference between one and two-way coupling depends on if the influence of the liquid phase onto the gas is considered with source terms (two-way coupling) or if it is neglected, so that the gaseous phase will not be perturbed by the particles (one-way coupling).


\begin{figure}[h!]
	\centering
	\includegraphics[scale=0.6]{./part1_numerical_approaches/figures_ch3/balachandar_disperse_phase_classification}
	\caption{Several numerical approaches to solve dispersed phase flows. Classification is done with respect to the liquid volume fraction (here defined as $\phi$) and to the Stokes number or, equivalently, to the ratio of largest to smallest length scales, which depend on the numerical resolution.  Source: \citeColor[balachandar_scaling_2009]}
	\label{fig:balachandar_numerical_methods_representation}
\end{figure}

In this chapter, numerical strategies to simulate dispersed phase flows are reviewed. Section \ref{sec:ch3_numerical_approaches_dispersed_phase} summarizes some of the available formalisms that can be chosen for solving a dispersed phase problem, with special emphasis on the lagrangian point-particle approach which is employed in this thesis. Then, section \ref{sec:ch3_state_art_lagrangian_injection} presents the state of the art on lagrangian methods to simulate fuel injection in MSFI systems, which is the starting point of the works presented in this thesis.

\section{Numerical approaches to model dispersed phase flows}
\label{sec:ch3_numerical_approaches_dispersed_phase}


%The different dispersed phase methodologies aim at resolving Eq. (\ref{eq:WBE_spray_representation}), either by approximating this equation to obtain the function $f$ or by making point-particles assumptions, as shown in Figure \ref{fig:balachandar_numerical_methods_representation}. This chapter makes a review of some of these strategies, with special emphasis on the lagrangian point-particle approach which is employed in this thesis.



\subsection{Statistical methods}

A spray composed of dispersed droplets can be characterized by means of a number density function $f$. This function depends on time $t$, space $\textbf{x}$, velocity $\textbf{v}$ and radius $r$, and can be expressed as follows (\textbf{ref:2000-subramanian}):

\begin{equation}
f \left( t, \textbf{x}, \textbf{v}, r \right) = \sum_{p=1}^{N_d} \delta \left( \textbf{x} - \textbf{x}_p \left( t \right) \right)  \delta \left( \textbf{v} - \textbf{v}_p \left( t \right) \right) \delta \left( \textbf{r} - \textbf{r}_p \left( t \right) \right)
\end{equation}

where $N_d$ is the number of droplets in the spray, and $\textbf{x}_p$, $\textbf{v}_p$ and $r_p$ are each individual particles position, velocity and radius respectively. The function $f$ contains all the necessary information to represent the spray. This function is governed by a William-Boltzmann Equation (\textbf{ref:williams-1958}, see Ref. 40 Murrone):

\begin{equation}
\label{eq:WBE_spray_representation}
\frac{\partial f}{\partial t} + \nabla_{\textbf{x}_p} \left( \textbf{v}_p f \right) + \nabla_{\textbf{v}_p} \left( \textbf{F}_p f \right) + \frac{\partial E_{S_p} f}{\partial S_p} + \frac{\partial E_{T_p} f}{\partial T_p}
\end{equation}

where $\textbf{x}_p$, $\textbf{v}_p$, $S_p$ and $T_p$ are each individual particle's position, velocity, surface and temperature, respectively; $F$ is the force acting on the particle, and $E_S$ and $E_T$ are respectively the exchange terms due to mass (evaporation rate) and energy (heat transfer). Statistical approaches aim at determining the function $f$ by solving/approximating Eq. (\ref{eq:WBE_spray_representation}). Stochastic methods to solve for the dispersed phase include the stochastic lagrangian approach of \textbf{ref:Dukowicz-1980} (see Vie 2010 or Cheneau 2019), eulerian methods considering a monodisperse (\textbf{ref:drew-passmann-1995}, see Vie 2010) or a polydisperse spray (\textbf{ref:lauren-massog-2001}, see Vie 2010), and quadrature methods (see Vie 2010). These methods are out of the scope of the present work and hence are not discussed here.



\subsection{Eulerian methods}

Eulerian methods, also called Euler-Euler formalism (EE), are widely used to model dispersed two-phase flows. The same grid is used for resolving both the carrier and dispersed phases. The carrier phase is solved from the Navier-Stokes equations ($\S$\ref{sec:ch2_governing_equations}) applied to the gas. For characterizing the dispersed phase, averaged properties and statistical tools are used to solve the Navier-Stokes equations. These methods present, however, difficulties when dealing with the polydispersion of the spray and when modeling the particles collisions and interactions with the walls \citepColor[garcia_developpement_2009].

To solve for the dispersed phase, a well-known eulerian formalism is the mesoscopic approach. The particles are described according to the kinetic theory of gases formulated by \textbf{ref:chapman-cowling-1970} (see Lancien thesis), so that mesoscopic variables are used to get averaged properties of the spray. In order to develop the equations for the dispersed phase, several assumptions need to be made \citepColor[lancien_etude_2018]:

\begin{enumerate}

\item The atomization process is complete, particles are perfectly spherical and non-deformable.

\item The only force exerted by the carrier phase on the droplets is drag.

\item Temperature is homogeneous inside each droplet.

\item Spray is dilute: $\alpha_l < 0.01$.

\item The interactions between droplets are neglected.

\item The spray is mono-disperse and mono-kinetic: at one point in time and space, the droplets all have the same diameter and velocity.

\item Similarly, at one point in time and space, all droplets have the same temperature.

%\item Random uncorrelated motion is neglected.

\end{enumerate}

With these assumptions, the dispersed phase is represented by the following equations based on statistical averages of the spray \citeColor[lancien_etude_2018]:

\begin{subequations}
\label{eq:EE_disperse_phase}
\begin{align}
\frac{\partial \overline{n}_l}{\partial t} + \nabla \left( \overline{n}_l \overline{\textbf{u}}_l \right)  &= 0 \\
\frac{\partial \rho_l \overline{\alpha}_l}{\partial t} + \nabla \left( \rho_l \overline{\alpha}_l \overline{\textbf{u}}_l \right) &= - \Gamma \\
\frac{\partial \rho_l \overline{\alpha}_l \overline{u}_{l,i}}{\partial t} + \nabla \left( \rho_l \overline{\alpha}_l \overline{\textbf{u}}_l \overline{u}_{l,i} \right) &= F_{d,i} - \overline{u}_{l,i} \Gamma  + \nabla \overline{\overline{\pmb{\tau}}}_l^{sgs} \\
\frac{\partial \rho_l \overline{\alpha}_l \overline{h}_l}{\partial t} + \nabla \left( \rho_l \overline{\alpha}_l \overline{\textbf{u}}_l  \overline{h}_l \right) &=  - \overline{h}_l  \Gamma + \overline{\Phi}
\end{align}
\end{subequations}

where $n_l$ is the number density of the particles, $\Gamma$ the evaporation rate, $F_{d}$ the drag force, $\overline{\overline{\pmb{\tau}}}_l^{sgs}$ the subgrid tensor of the liquid phase, $\overline{h}_l$ the liquid enthalpy and $\overline{\Phi}$ the conduction flux term. The exchange between phases is taken into account by the right hand sides of the previous expressions. As it is observed, multiphysics phenomena such as mass exchange due to evaporation and energy transfer can be taken into account with the corresponding exchange terms.  This is one of the advantage of the dispersed phase modelling of two-phase flows as opposed to the fully-resolved atomization methods presented in Chapter \ref{ch2:numerical_methods_resolved_atomization}. 

\subsection{Lagrangian point particle representation}
\label{sec:ch3_EL_formalisms}

Another well-know methodology to simulate dispersed phases systems, broadly employed in aerospace applications, is the lagrangian point particle (LPP) representation (see Figure \ref{fig:balachandar_numerical_methods_representation}). In lagrangian formalisms, the gas is resolved in the main eulerian grid, but the particles conforming the dispersed phase are tracked individually and represented by a different set of equations. Since the dispersed phase is not resolved in the main grid, these method is usually referred as Euler-Lagrange (EL) formalism.


The Lagrangian description (EL) does not use an averaging process in the grid for the fluid phase, but tracks each fluid particle individually. Every droplet is represented by its own equations which are not solved in the main grid (i.e. the Eulerian grid used to represent the carrier phase). This can make convergence difficult and hinders the introduction of parallelism techniques. However, the resulting system of equations is robust, the time per iteration is usually lower than for the EE description, and the drop-drop and drop-wall interactions are easier to model.

When considering droplets immersed in a gas, it is necessary to take into account mechanics of particle motion. A useful definition to study the relative influence between the droplet and the gas is the \textbf{Stokes number}, which is defined as the ratio between the characteristic time of the particle $\tau_p$ and the frequency of the flow fluctuations $f_{turb}$ around an obstacle \citepColor[koch_spray_2019]:

\begin{equation}
St = \tau_p f_{turb} = \frac{\tau_p \widetilde{\boldsymbol{u}}_g}{d_p}
\end{equation}

where $\widetilde{\boldsymbol{u}}_g$ is the gas velocity at the position of the particle $p$ assuming that the flow field is locally undisturbed by the presence of that particle. The Stokes number indicates how the droplet responds to fluctuations of the gas flow. If $St << 1$, the droplet will follow perfectly the fluctuations of the gas flow. On the contrary, if $St >> 1$ the droplets will rather neglect the flow trajectories and follow a ballistic trajectory. $\tau_p$ is obtained from the following expression (\textbf{CHECK THIS OUT}):


\begin{equation}
\tau_p = \frac{d_p^2 \rho_p}{18 \mu_\mathrm{g}} =  \frac{4}{3} \frac{\rho_p}{\rho_\mathrm{g}} \frac{d_p}{C_D | \boldsymbol{v}_r |} = \frac{4}{3} \frac{\rho_p d_{p^2}}{\mu_g Re C_D}
\end{equation}

For the description in the Lagrangian framework, each particle will be modelled according to the \textbf{Discrete Particle Simulations} (DPS) approach. DPS assumes each particle to be completely spherical and robust. With these assumptions, the \textbf{dynamic equations} for a particle $p$ in the spatial direction $i$ are:

\begin{subequations}
\label{eq:TPF_lagrange_dynamic_eqs}
\begin{align}
\frac{d \boldsymbol{x}_p}{d t} &= \boldsymbol{u}_p \\
\frac{d \boldsymbol{u}_p}{d t} &= \underbrace{ - \frac{3}{4} \frac{\rho_\mathrm{g}}{\rho_p} \frac{C_D}{d_p} | \boldsymbol{v}_r | \boldsymbol{v}_r }_{\text{Drag}\atop\text{term}}  + \underbrace{ \left( 1 - \frac{\rho_\mathrm{g}}{\rho_p} \right) \boldsymbol{g} }_{\text{Gravity}\atop\text{term}} 
\end{align}
\end{subequations}

% \frac{d u_{p,i}}{d t} &= \underbrace{ - \frac{3}{4} \frac{\rho_\mathrm{g}}{\rho_p} \frac{C_D}{d_p} | \boldsymbol{v}_r | v_{r,i} }_{\text{Drag}\atop\text{term}}  + \underbrace{ \left( 1 - \frac{\rho_\mathrm{g}}{\rho_p} \right) g_i }}_{\text{Gravity}\atop\text{term}} 

where $\rho_\mathrm{g}$ and $\rho_p$ are the gas and liquid particle densities respectively, $C_D$ is a drag coefficient, $d_p$ is the particle diameter, $\boldsymbol{v}_r$ is the relative velocity of the particle with respect to the airflow and $g$ is the gravity term. Equation (\ref{eq:TPF_lagrange_dynamic_eqs}b) is the momentum equation of the droplet and contains the contributions of the two forces considered: the drag force and the gravitational forces. The relative velocity is given by $\boldsymbol{v}_r = \boldsymbol{u}_p - \widetilde{\boldsymbol{u}}_g$ 

The drag coefficient $C_D$ is given by the following expression:

\begin{equation}
\label{eq:Re_CD_droplet}
C_D =
\left\{
    \begin{split}
     \frac{24}{Re} \left( 1 + 0.15 Re^{0.687} \right)\,\,\mathrm{if}\,\,Re < 1000 \\ 
    0.44\,\,\mathrm{if}\,\,Re \geq 1000 
    \end{split}
\right.
\end{equation}

where $Re = \rho_\mathrm{g} | \boldsymbol{v_r} | d_p / \mu_\mathrm{g} $. This expression was obtained experimentally by \citeColor[schiller_drag_1935] with the assumptions that the particles are perfectly spherical and rigid.  

%These assumptions also allow for the definition of the characteristic time of a particle $p$:

%\begin{equation}
%\tau_p = \frac{d_p^2 \rho_p}{18 \mu_\mathrm{g}} =  \frac{4}{3} \frac{\rho_p}{\rho_\mathrm{g}} \frac{d_p}{C_D | \boldsymbol{v_r} |} 
%\end{equation}

Besides the dynamic equations, it is also necessary to account for the \textbf{phase transition} of the liquid dispersed phase into gaseous phase. 

\begin{subequations}
\label{eq:TPF_lagrange_phasetransition_eqs}
\begin{align}
\frac{d m_p}{d t} &= \dot{m}_p = \Gamma \\
\frac{d h_p}{d t} &= \Phi_p
\end{align}
\end{subequations}

Equation (\ref{eq:TPF_lagrange_phasetransition_eqs}a) accounts for the \textbf{evaporation} process. If the uniform temperature model is used, then (\ref{eq:TPF_lagrange_phasetransition_eqs}) is equal to (\ref{eq:evaporation_mass_flow_rate}). Equation (\ref{eq:TPF_lagrange_phasetransition_eqs}b) accounts for the energy transfer. By considering the effects of the \textbf{conduction} and the \textbf{evaporation} processes in terms of energetic balance, $\Phi_p$ can be extended as follows:

\begin{equation}
\Phi_p = m_p C_{P_\mathrm{g}} \frac{d T_p}{d t} = \underbrace{ h_p \pi d_p^2 \left( T_\mathrm{g} - T_p \right) }_{\text{Conduction}\atop\text{term}} - \underbrace{ \dot{m}_p  \Delta h_v }_{\text{Evaporation}\atop\text{term}}
\end{equation}

where $C_{P_\mathrm{g}}$ is the gas specific heat capacity at constant pressure, $T_p$ is the particle temperature, $T_\mathrm{g}$ is the gas temperature, $h_p$ is the film coefficient of the gas at the particle surface and $\Delta h_v$ is the latent heat of vaporization of the particle. 

By considering Equations (\ref{eq:TPF_lagrange_dynamic_eqs}) and (\ref{eq:TPF_lagrange_phasetransition_eqs}) with all the simplifications stated, the final set of equations for the Lagrangian dispersed phase approach is:

\begin{subequations}
\begin{align}
\frac{d x_{p,i}}{d t} &= u_{p,i} \\
\frac{d u_{p,i}}{d t} &= - \frac{u_{p,i} - \widetilde{u}_{g,i} }{\tau_p} + \left( 1 - \frac{\rho_\mathrm{g}}{\rho_p} \right) g_i \\
\frac{d m_p}{d t} &= \Gamma \\
m_p C_{P_\mathrm{g}} \frac{d T_p}{d t} &= h_p \pi d_p^2 \left( T_\mathrm{g} - T_p \right) - \dot{m}_p  \Delta h_v
\end{align}
\end{subequations}



\section{Lagrangian injection models for multipoint injection}
\label{sec:ch3_state_art_lagrangian_injection}

A classification of models for liquid injection in dispersed phase computations is proposed in Figure \ref{fig:state_art_injection}. Each category has the following advantages and disadvantages (see presentation 2020 12 15):

\begin{itemize}

	\item \textbf{Based on empirical laws}. \textcolor{green}{Advantage}: embedded physics. \textcolor{red}{Disadvantage}: restricted to geometry and operating conditions of development.
	
	\item \textbf{Injection of fully developed spray}. \textcolor{green}{Advantage}: droplets size distributions directly imposed. \textcolor{red}{Disadvantage}: Primary atomization neglected.
	
	\item \textbf{Effect of dense spray in gaseous phase}. \textcolor{green}{Advantage}: in JICF, taking into account blockage effect. \textcolor{red}{Disadvantage}: None.
	
	\item \textbf{Addition of secondary atomization model}. Good if atomziation not present.
	
	\item \textbf{Reference spray learning}. So far, and to our knowledge, there are no available models that can use a reference spray to learn boundary conditions (droplet sizes, velocity distributions, spatial location of injectors) for dispersed phase computations. \textcolor{green}{Advantage}: Applicable to generic injectors and 
	wide range of operating conditions.  \textcolor{red}{Disadvantage}: Need of detailed spray characterisation from simulations or experiments.

\end{itemize}

%\begin{figure}[h!]
%	\centering
%	\includegraphics[scale=0.5]{./part1_numerical_approaches/figures_ch3/state_art_lagrangian}
%	\caption{Classification of lagrangian injection models}
%	\label{fig:state_art_injection}
%\end{figure}

\begin{figure}[h!]	
	\centering
	\includeinkscape[inkscapelatex=true,scale=0.75]{./part1_numerical_approaches/figures_ch3/state_art_lagrangian}
	\caption{Classification of lagrangian injection models}
	\label{fig:state_art_injection}
\end{figure}

\subsection{Airblast spray}

Basically Chaussonnet.

\subsection{Hollow cone spray}

Basically FIMUR.

\subsection{Liquid jet in crossflow}

Aqui viene el arsenal.