\chapter{Spray learning from resolved atomization simulations of BIMER}
	\label{ch8:bimer_resolved_atomization}

Steps to follow (for this chapter):

\begin{enumerate}

	\item Intro: make it atractive. Specify that we are gonna simulate just one multipoint injector.
	
	\item Numerical setup: specify that :

		\begin{itemize}
		
			\item Fine mesh from the previous chapter is used
			
			\item One multipoint injector is going to be simulated (make zoom, nice pictures and graphs showing it)
			
			\item The simulation detailed in $\S$\ref{ch7:BIMER_application_point} is used as initial solution
			
			\item Describe the operating point ($q$, $We$). Mention how the bulk gas velocity has been obtained.
		
		\end{itemize}
	
\end{enumerate}


\section{Introduction}



\section{Numerical setup}

\begin{table}[!h]
\centering
\caption{Operating point to perform gaseous and two-phase simulations tested by \citeColor[renaud_high-speed_2015]}
\begin{tabular}{|c|c|c|c|}
\hline
\multicolumn{4}{|c|}{\textbf{Air properties}} \\
\hline
$\dot{m}_g$ [g s$^{-1}$] & $T_g$ [K] & $\rho_g$ [kg m$^{-3}$]  & $\mu_g$ [Pa s]  \\
\hline
43.1 & 433 & 0.816382 & $2.3911 \cdot 10^{-5}$ \\
\hline
\hline
\multicolumn{4}{|c|}{\textbf{Liquid properties}} \\
\hline
$\dot{m}_l$ [g s$^{-1}$] & $\rho_l$ [kg m$^{-3}]$   & $\mu_l$ [Pa s]   & $\sigma$ [N/m]   \\
\hline
1.64 & 750 & $1.36 \cdot 10^{-3}$ & $25.35 \cdot 10^{-3}$ \\
\hline
\hline
\multicolumn{4}{|c|}{\textbf{Burner staging}} \\
\hline
$\alpha$ [$\%$] & $\dot{m}_{l,pilot}$ & $\dot{m}_{l,takeoff}$ & \\
\hline
15 & 0.25 & 1.39 & \\
\hline
\end{tabular}
\label{tab:liquid_operating_point_Renaud}
\end{table}


\section{Liquid injection through one multipoint hole}

\subsection{Determination of bulk gas velocity}

Two methods to determine bulk gas velocity:

\begin{itemize}

	\item According to \citeColor[barbosa_etude_2008], a $80 \%$ of the air flow rate goes through the take-off stage and $20 \%$ goes through the take-off one. The study of \citeColor[renaud_high-speed_2015] shows that the percentage through the pilot is actually $13.5 \%$, hence $86.5 \%$ goes through the take-off. Considering the application point in Table \ref{tab:gaseous_operating_points_BIMER}, this makes a total of $\dot{m}_{a,take-off} = 37.2815 ~ $ g s$^{-1}$, which in terms of flow rate is $Q = \dot{m} / \rho_g = 0.04567 $ m$^{3}$ s$^{-1}$. Considering that this flow rate is split through 20 vanes, then per canal the flow rate is $Q_a = 2.2833 \cdot 10^{-3}$ m$^{3}$ s$^{-1}$. With an area of 6 x 10 mm$^2$ (see Fig. \ref{fig:gas_injection_area_multipoint}), the estimated bulk velocity is $u_g = Q / A = 38 ~ m/s$
	
	\item Nah

\end{itemize}

The momentum flux ratio $q$ can now be calculated:

\begin{equation}
q = \frac{\rho_l u_l^2}{\rho_g u_g^2} = \frac{750 \cdot 2.6^2}{0.816382 \cdot 38.05^2} = 4.36
\end{equation}

And also the $We$ number based on the gaseous flow:

\begin{equation}
We_g = \frac{\rho_g d_{inj} u_g^2}{\sigma} = \frac{0.816382 \cdot 0.3 mm \cdot 38.05^2}{25.35 10^{-3}} \approx 14
\end{equation}

\begin{table}[!h]
\centering
\caption{JICF operating points studied}
\begin{tabular}{lccc}
\thickhline
\textbf{Parameter} & \textbf{Symbol} & \textbf{Units} &  Operating point \\
\thickhline
Nozzle diameter & $d_\mathrm{inj}$ & mm & 0.3 \\
%\hline
Gas bulk velocity & $u_g$ & m s$^{-1}$ & 38 \\
%\hline
Liquid bulk velocity & $u_l$ & m s$^{-1}$ & 2.6  \\
%\hline
Liquid flow rate & $Q_l$ & mm$^3$ s$^{-1}$ &   \\
%\hline
Gas density & $\rho_g$ & kg m$^{-3}$ &  \\
%\hline
Liquid density & $\rho_l$ & kg m$^{-3}$ &  \\
%\hline
Gas viscosity & $\mu_g$ & kg m$^{-1}$ s$^{-1}$ &  \\
%\hline
Liquid viscosity & $\mu_l$ & kg m$^{-1}$ s$^{-1}$ &   \\
%\hline
Surface tension & $\sigma$ & kg s$^{-2}$ &  0.022  \\
\thickhline
Gas Reynolds number & $Re_g$ & - & \\
%\hline
Liquid Reynolds number & $Re_l$ & - &  \\
%\hline
Momentum ratio & $q$ & - & 4  \\
%\hline
Gas Weber number & $We_g$ & - &  \\
%\hline
Relative Weber number & $We_\mathrm{rel}$ & - & \\
%\hline
Aerodynamic Weber number & $We_\mathrm{aero}$ & - & 14 \\
%\hline
Ohnesorge number & $Oh $ & - & \\
%\hline
Density ratio & $r$ & - & \\
%\hline
\thickhline
\end{tabular}
\label{tab:bimer_sps_operating_point}
\end{table}

\begin{table}[!h]
\centering
\caption{Operating point}
\begin{tabular}{cccc}
\thickhline
$u_g$ [m s$^{-1}$] &  38 \\
\hline
$u_l$ [m s$^{-1}$] &  2.6 \\
\hline
\hline
$q$ & 4 \\ %4.3 \\
\hline
$We_g$ & 14 \\
\hline
\end{tabular}
\label{tab:bimer_sps_operating_point}
\end{table}

\begin{table}[!h]
\centering
\caption{Operating point}
\begin{tabular}{cccc}
\thickhline
$u_g$ [m s$^{-1}$] &  $u_l$ [m s$^{-1}$] & $q$ &  $We_g$  \\
\hline
38 &  2.6 & 4 & 15 \\
\thickhline
\end{tabular}
\label{tab:bimer_sps_operating_point}
\end{table}



%\begin{figure}[h!]	
%	\centering
%	\includeinkscape[inkscapelatex=false,scale=0.75]{./part1_numerical_approaches/figures_ch3/gas_injection_area_multipoint}
%	\caption{Area to calculate bulk gas velocity}
%	\label{fig:gas_injection_area_multipoint}
%\end{figure}

\subsection{Characteristic times}

The characteristic time scale by \citeColor[ranger_aerodynamics_1968] can be used:

\begin{equation}
\tau_c = \sqrt{\frac{\rho_l}{\rho_g}} \frac{d_\mathrm{inj}}{u_g} = \sqrt{\frac{750}{0.816382 }} \frac{0.3 ~\mathrm{mm}}{38} = 0.24 ~\mathrm{ms}
\end{equation}

We can also define as in Eq. (\ref{eq:jicf_tau_injector}):

\begin{equation}
\tau_\mathrm{ph} = \frac{d_\mathrm{inj}}{u_l} = \frac{0.3 ~\mathrm{mm}}{2.6} = 0.12 ~\mathrm{ms}
\end{equation}

%According to \textbf{2016 Eckel}, the velocity is not the gaseous one but the relative:
%
%\begin{equation}
%\tau_c = \sqrt{\frac{\rho_l}{\rho_g}} \frac{d_\mathrm{inj}}{u_\mathrm{rel}} = \sqrt{\frac{750}{0.816382 }} \frac{0.3 ~mm}{38 - 2.6} = 0.26
%\end{equation}
%
%which is actually very similar.

\section{Injectors learning}

\section{Conclusion}
