\chapter{Spray learning from resolved atomization simulations of BIMER}
	\label{ch8:bimer_resolved_atomization}

\section{Introduction}

\section{Numerical setup}

\begin{table}[!h]
\centering
\caption{Operating point to perform gaseous and two-phase simulations tested by \citeColor[renaud_high-speed_2015]}
\begin{tabular}{|c|c|c|c|}
\hline
\multicolumn{4}{|c|}{\textbf{Air properties}} \\
\hline
$\dot{m}_g$ [g s$^{-1}$] & $T_g$ [K] & $\rho_g$ [kg m$^{-3}$]  & $\mu_g$ [Pa s]  \\
\hline
43.1 & 433 & 0.816382 & $2.3911 \cdot 10^{-5}$ \\
\hline
\hline
\multicolumn{4}{|c|}{\textbf{Liquid properties}} \\
\hline
$\dot{m}_l$ [g s$^{-1}$] & $\rho_l$ [kg m$^{-3}]$   & $\mu_l$ [Pa s]   & $\sigma$ [N/m]   \\
\hline
1.64 & 750 & $1.36 \cdot 10^{-3}$ & $25.35 \cdot 10^{-3}$ \\
\hline
\hline
\multicolumn{4}{|c|}{\textbf{Burner staging}} \\
\hline
$\alpha$ [$\%$] & $\dot{m}_{l,pilot}$ & $\dot{m}_{l,takeoff}$ & \\
\hline
15 & 0.25 & 1.39 & \\
\hline
\end{tabular}
\label{tab:liquid_operating_point_Renaud}
\end{table}


\section{Liquid injection through one multipoint hole}

\subsection{Determination of bulk gas velocity}

Two methods to determine bulk gas velocity:

\begin{itemize}

	\item According to \textbf{ref:Barbosa-2008}, a $80 \%$ of the air flow rate goes through the take-off stage and $20 \%$ goes through the take-off one. The study of \citeColor[renaud_high-speed_2015] shows that the percentage through the pilot is actually $13.5 \%$, hence $86.5 \%$ goes through the take-off. Considering the application point in Table \ref{tab:gaseous_operating_points_BIMER}, this makes a total of $\dot{m}_{a,take-off} = 37.2815 ~ $ g s$^{-1}$, which in terms of flow rate is $Q = \dot{m} / \rho_g = 0.04567 $ m$^{3}$ s$^{-1}$. Considering that this flow rate is split through 20 vanes, then per canal the flow rate is $Q_a = 2.2833 \cdot 10^{-3}$ m$^{3}$ s$^{-1}$. With an area of 6 x 10 mm$^2$ (see Fig. \ref{fig:gas_injection_area_multipoint}), the estimated bulk velocity is $u_g = Q / A = 38 ~ m/s$
	
	\item Nah

\end{itemize}

The momentum flux ratio $q$ can now be calculated:

\begin{equation}
q = \frac{\rho_l u_l^2}{\rho_g u_g^2} = \frac{750 \cdot 2.6^2}{0.816382 \cdot 38.05^2} = 4.36
\end{equation}

And also the $We$ number based on the gaseous flow:

\begin{equation}
We_g = \frac{\rho_g d_{inj} u_g^2}{\sigma} = \frac{0.816382 \cdot 0.3 mm \cdot 38.05^2}{25.35 10^{-3}} \approx 14
\end{equation}


\begin{table}[!h]
\centering
\caption{Operating point}
\begin{tabular}{|c|c|}
\hline
$u_g$ [m s$^{-1}$] &  38 \\
\hline
$u_l$ [m s$^{-1}$] &  2.6 \\
\hline
\hline
$q$ & 4 \\ %4.3 \\
\hline
$We_g$ & 14 \\
\hline
\end{tabular}
\label{tab:bimer_sps_operating_point}
\end{table}


%\begin{figure}[h!]	
%	\centering
%	\includeinkscape[inkscapelatex=false,scale=0.75]{./part1_numerical_approaches/figures_ch3/gas_injection_area_multipoint}
%	\caption{Area to calculate bulk gas velocity}
%	\label{fig:gas_injection_area_multipoint}
%\end{figure}

\subsection{Characteristic times}

The characteristic time scale by \textbf{ref:ranger-nicholls(1969} (see 2003 Sallam) can be used:

\begin{equation}
\tau_c = \sqrt{\frac{\rho_l}{\rho_g}} \frac{d_\mathrm{inj}}{u_g} = \sqrt{\frac{750}{0.816382 }} \frac{0.3 ~mm}{38} = 0.24 ~ms
\end{equation}

%According to \textbf{2016 Eckel}, the velocity is not the gaseous one but the relative:
%
%\begin{equation}
%\tau_c = \sqrt{\frac{\rho_l}{\rho_g}} \frac{d_\mathrm{inj}}{u_\mathrm{rel}} = \sqrt{\frac{750}{0.816382 }} \frac{0.3 ~mm}{38 - 2.6} = 0.26
%\end{equation}
%
%which is actually very similar.

\section{Injectors learning}

\section{Conclusion}
