\chapter{BIMER LGS: extra stuff (remove on due time)}
	\label{ch9:BIMER_lagrangian_extra}


\section{Boundary condition for liquid phase}
\label{sec:ch9_BIMER_BCs_for_liquid_phase}


\subsection{Multipoint stage injection}


\subsection{Pilot stage injection}

The operating point simulated injects fuel through both a take-off stage (which has been modelled with the SLI) and a pilot stage. Since pilot stage has not been simulated with the methodology developed in this thesis, another approach must be employed. Given that the pilot of BIMER injects fuel following a hollow cone configuration, the LISA model introduced in $\S$\ref{subsec:ch3_hollow_cone_spray} will be used \citepColor[guedot_developpement_2015].   The input parameters for the LISA model to inject a hollow cone spray are summarized in Table \ref{tab:LISA_model_parameters}. For the mean opening angle, a value $\theta_s = 30 \degree$ is taken from experiments \citepColor[renaud_high-speed_2015]. Regarding the droplet's diameter, previous studies using lagrangian approaches on the same configuration have introduced directly droplets size distributions extracted from experimental data \citepColor[mesquita_large_2018]. In this case, since for the operating condition simulated there is not experimental size distributions available, droplets injected will have a constant diameter given by the following experimental correlation \citepColor[lefebvre_atomization_2017]:

\begin{equation}
SMD = 2.25 \left( \sigma \dot{m}_f \mu_l \right)^{0.25} \rho_g^{-0.25}  \Delta P^{-0.5}
\end{equation}

where $\Delta P = 2.6$ MPa is the pressure drop in the pilot nozzle \citeColor[renaud_high-speed_2015]. Applying the previous correlation with the magnitudes given in Table \ref{tab:liquid_operating_point_Renaud} results in a diameter $SMD = 15~\mu$m imposed to the pilot cones particles. Such particles will later break due to the action of the secondary atomization model. The breakup model is constant in all the simulations and therefore is identical for both the take-off and pilot stages, hence the Gorokhovski model with the constants defined in Table \ref{tab:BIMER_SLI_fixed_model_parameters} is applied to the pilot droplets. 

 %this correlation provides a SMD of $15 \mu m$ (\textbf{CHECK THIS}).

% dP value is tiven in Renaud p. 24, or 46 PDF

\begin{table}[!h]
\centering
\caption{LISA model setup for pilot injection}
\begin{tabular}{ccc}
\thickhline
\textbf{Parameter} & \textbf{Units} &  \textbf{Value} \\
\thickhline
Mass flow rate $\dot{m}$ & g s$^{-1}$ & 0.25 \\
%\hline
Injector radius $R_0$ & mm & 0.125 \\
%\hline
Mean angle $\overline{\theta}_s$ & $\degree$ & 30  \\
SMD & $\mu$m & 15 \\
\thickhline
\end{tabular}
\label{tab:LISA_model_parameters}
\end{table}

\subsubsection*{LISA model for injection (REMOVE WHEN SIMUS READY)}

For performing pilot injection, the LISA model available in YALES2 is used \citepColor[guedot_developpement_2015]. 

\begin{equation}
X = \frac{A_a}{A_0} = \left( \frac{R_a}{R_0} \right)^2 = \frac{\sin^2 \gamma_s}{1 + \cos^2 \gamma_s}
\end{equation}

where $R_a$ is the minimum injection radius, $R_0$ the maximum, and $\gamma_s$ is the mean injection angle. The inputs to the model are $\gamma_s$ and $R_0$, so from the previous equation $X$ can be obtained and the radius $R_a$ can be solved:

\begin{equation}
R_a^2 = R_0^2 X
\end{equation}

The axial velocity imposed to the particles $u_x$ is obtained through the following expression:

\begin{equation}
u_x = \frac{\dot{m}}{\rho_l \pi \left( R_0^2 - R_a^2 \right)} = \frac{\dot{m}}{\rho_l \pi R_0^2 \left( 1 - X^2 \right)} 
\end{equation}

\textbf{OJO}: in Renaud's application case ($\dot{m} = 0.25 ~ g/s$, see Table \ref{tab:LISA_model_parameters}), taking $R_0 = 0.125 ~mm$ gives an axial velocity of $u_x = 7.9 ~ m/s$. This created an almost-point injection, as it can be seen in the simulations. If we take $R_0 = 1.5 ~mm$ (the radius of the hollow cone patch), droplets are injected along all the hollow cone patch. However, the axial velocity imposed is $u_x = 0.05 ~m/s$. So we'll go towards the large radius.

%\begin{table}[!h]
%\centering
%\caption{LISA model setup for pilot injection}
%\begin{tabular}{|c|c|}
%\hline
%\textbf{Parameter} & \textbf{Value} \\
%\hline
%Mass flow rate $\dot{m} [g s^{-1}]$ & 0.25 \\
%\hline
%Injector radius $R_0 [mm]$ & 0.125 \\
%\hline
%Mean angle $\overline{\theta} [\degree]$ & 40 \\
%\hline
%\end{tabular}
%\label{tab:LISA_model_parameters}
%\end{table}





\section{Extrapolation of injectors to rest of multipoint holes}

The 

\subsection{Injectors geometry}

\begin{equation}
\boldsymbol{x}_0 =  \begin{pmatrix} - 38.5 ~\mathrm{mm} \\ r \cos \alpha_0 \\ r \sin \alpha_0 \end{pmatrix}
\end{equation}

\begin{equation}
\boldsymbol{x}_i =  \begin{pmatrix} - 38.5 ~\mathrm{mm} \\ r \cos \alpha_i \\ r \sin \alpha_i \end{pmatrix}
\end{equation}


\subsection{General procedure}

\begin{enumerate}

	\item Obtain parameters for SLI of injector 0 (baseline parameters):
	
	\begin{equation}
	\alpha_0  ~~ ; ~~ \boldsymbol{n}_0 ~~ ; ~~ \theta_0 = 90 - \alpha_0 - atan \left( \frac{n_y}{n_z} \right) ~~ ; 
	\end{equation}

	\item Get parameters for SLI of injector $i$ from baseline:
	
	\begin{equation}
	\alpha_i = \alpha_0 - i \Delta \alpha 
	\end{equation}
	
	\begin{equation}
	\boldsymbol{n}_i = 
	\end{equation}
	
	\begin{equation}
	\theta_1 = \theta_0
	\end{equation}
	

\end{enumerate}

\subsection{Definition of coordinate systems and operations}

The global coordinate system is:

\begin{equation}
\boldsymbol{x} =  \begin{pmatrix} x \\ y \\ z \end{pmatrix}
\end{equation}

The local (crossflow) coordinate system is:

\begin{equation}
\boldsymbol{x}^{cr} = \begin{pmatrix} x^c \\ y^c \\ z^c \end{pmatrix}
\end{equation}

with the following equivalences between local and global systems :

\begin{equation}
\boldsymbol{x}^c = \boldsymbol{n}  ~~~~ ; ~~~~ \boldsymbol{z}^c = \boldsymbol{x}  ~~~~ ; ~~~~ \boldsymbol{y}^c =  \boldsymbol{z}^c \times \boldsymbol{x}^c
\end{equation}

where the rotation matrix being:

\begin{equation}
\boldsymbol{R} = \begin{pmatrix} \boldsymbol{x}^{c^T} \\ \boldsymbol{y}^{c^T} \\ \boldsymbol{z}^{c^T} \end{pmatrix}
\end{equation}

More elegantly expresses:

\begin{equation}
\boldsymbol{R} = \begin{pmatrix} x^c_x & x^c_y & x^c_z \\ y^c_x & y^c_y & y^c_z \\ z^c_x & z^c_y & z^c_z \end{pmatrix}
\end{equation}

%\begin{equation}
%\boldsymbol{x} =  \begin{pmatrix} 1 & 2 & -3 \\ 4 & 0 & 1 \end{pmatrix}
%\end{equation}


Transformation for droplet locations is translation + rotation

\begin{equation}
\boldsymbol{x}^c_\mathrm{dr} = \boldsymbol{R} \left( \boldsymbol{x}_\mathrm{dr} -  \boldsymbol{x}_0 \right)
\end{equation}

For droplet velocities, transformation is only rotation:

\begin{equation}
\boldsymbol{u}^c_\mathrm{dr} = \boldsymbol{R} \boldsymbol{u}_\mathrm{dr}
\end{equation}

Inverse transform then:

\begin{equation}
\boldsymbol{x}_\mathrm{inj} = \boldsymbol{x}_0 + \boldsymbol{R}^{-1} \boldsymbol{x}_\mathrm{inj}^c
\end{equation}

\begin{equation}
\boldsymbol{u}_\mathrm{inj} = \boldsymbol{R}^{-1} \boldsymbol{u}_\mathrm{inj}^c
\end{equation}

\section{Towards reactive simulations}


